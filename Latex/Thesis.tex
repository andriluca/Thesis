% A LaTeX template for MSc Thesis submissions to 
% Politecnico di Milano (PoliMi) - School of Industrial and Information Engineering
%
% S. Bonetti, A. Gruttadauria, G. Mescolini, A. Zingaro
% e-mail: template-tesi-ingind@polimi.it
%
% Last Revision: October 2021
%
% Copyright 2021 Politecnico di Milano, Italy. NC-BY

\documentclass{Configuration_Files/PoliMi3i_thesis}
% A LaTeX template for EXECUTIVE SUMMARY of the MSc Thesis submissions to 
% Politecnico di Milano (PoliMi) - School of Industrial and Information Engineering
%
% P. F. Antonietti, S. Bonetti, A. Gruttadauria, G. Mescolini, A. Zingaro
% e-mail: template-tesi-ingind@polimi.it
%
% Last Revision: October 2021
%
% Copyright 2021 Politecnico di Milano, Italy. Inc. All rights reserved.

\documentclass[11pt,a4paper,twocolumn]{article}

%------------------------------------------------------------------------------
%	REQUIRED PACKAGES AND  CONFIGURATIONS
%------------------------------------------------------------------------------
% PACKAGES FOR TITLES
\usepackage{titlesec}
\usepackage{color}

% PACKAGES FOR LANGUAGE AND FONT
\usepackage[utf8]{inputenc}
\usepackage[english]{babel}
\usepackage[T1]{fontenc} % Font encoding

% PACKAGES FOR IMAGES
\usepackage{graphicx}
\graphicspath{{Images/}} % Path for images' folder
\usepackage{eso-pic} % For the background picture on the title page
\usepackage{subfig} % Numbered and caption subfigures using \subfloat
\usepackage{caption} % Coloured captions
\usepackage{transparent}

% STANDARD MATH PACKAGES
\usepackage{amsmath}
\usepackage{amsthm}
\usepackage{bm}
\usepackage[overload]{empheq}  % For braced-style systems of equations

% PACKAGES FOR TABLES
\usepackage{tabularx}
\usepackage{longtable} % tables that can span several pages
\usepackage{colortbl}

% PACKAGES FOR ALGORITHMS (PSEUDO-CODE)
\usepackage{algorithm}
\usepackage{algorithmic}

% PACKAGES FOR REFERENCES & BIBLIOGRAPHY
\usepackage[colorlinks=true,linkcolor=black,anchorcolor=black,citecolor=black,filecolor=black,menucolor=black,runcolor=black,urlcolor=black]{hyperref} % Adds clickable links at references
\usepackage{cleveref}
\usepackage[backend=biber,style=numeric]{biblatex}
\addbibresource{bibliography.bib}
% \usepackage[square, numbers, sort&compress]{natbib} % Square brackets, citing references with numbers, citations sorted by appearance in the text and compressed
% \bibliographystyle{plain} % You may use a different style adapted to your field

% PACKAGES FOR THE APPENDIX
\usepackage{appendix}

% PACKAGES FOR ITEMIZE & ENUMERATES 
\usepackage{enumitem}

% OTHER PACKAGES
\usepackage{amsthm,thmtools,xcolor} % Coloured "Theorem"
\usepackage{comment} % Comment part of code
\usepackage{fancyhdr} % Fancy headers and footers
\usepackage{lipsum} % Insert dummy text
\usepackage{tcolorbox} % Create coloured boxes (e.g. the one for the key-words)
\usepackage{stfloats} % Correct position of the tables

%-------------------------------------------------------------------------
%	NEW COMMANDS DEFINED
%-------------------------------------------------------------------------
% EXAMPLES OF NEW COMMANDS -> here you see how to define new commands
\newcommand{\bea}{\begin{eqnarray}} % Shortcut for equation arrays
\newcommand{\eea}{\end{eqnarray}}
\newcommand{\e}[1]{\times 10^{#1}}  % Powers of 10 notation
\newcommand{\mathbbm}[1]{\text{\usefont{U}{bbm}{m}{n}#1}} % From mathbbm.sty
\newcommand{\pdev}[2]{\frac{\partial#1}{\partial#2}}
% NB: you can also override some existing commands with the keyword \renewcommand

%----------------------------------------------------------------------------
%	ADD YOUR PACKAGES (be careful of package interaction)
%----------------------------------------------------------------------------
\usepackage{siunitx}
\usepackage[siunitx]{circuitikz}
\usepackage{stfloats}
\usepackage{tikz}
\usetikzlibrary{shapes.geometric, arrows, backgrounds, fit}
\ctikzset{bipoles/length=1.0cm}
\usepackage[autoload=true,
theme=grayscale-plain]{jlcode}

% To use with LuaLatex or XeTeX
% \usepackage{fontspec}

%----------------------------------------------------------------------------
%	ADD YOUR DEFINITIONS AND COMMANDS (be careful of existing commands)
%----------------------------------------------------------------------------
\lstdefinestyle{juliacode}{
  language=Julia, 
  showstringspaces=false,
  numbers=left,
  % stepnumber=1,
  numberstyle=\tiny,
  stepnumber=2,
  % numbersep=10pt,
  % Comment the following three lines to remove colors
  keywordstyle=\color{blue},
  commentstyle=\color{gray},
  identifierstyle=\color{purple!80},
  columns=fullflexible,
  keepspaces=true
}

\lstnewenvironment{juliacode}{\lstset{style=juliacode}}{}

%% To use with LuaLatex or XeTeX
% \newfontfamily \JuliaMono {JuliaMono-Light.ttf}[
%     Path      = ./,
%     Extension = .ttf
%     ]
% \newfontface \JuliaMonoLight{JuliaMono-Light}
% \setmonofont{JuliaMono-Light}[ Contextuals=Alternate ]

\tikzstyle{startstop} = [rectangle, rounded corners, minimum width=3cm, text width=3cm, minimum height=1cm,text centered, draw=black, fill=red!30]
\tikzstyle{process}   = [rectangle, minimum width=2cm, text width=3cm, minimum height=1cm, text centered, draw=black, fill=orange!30]
% \tikzstyle{process}   = [rectangle, draw=black, text width=3cm, fill=orange!30]
\tikzstyle{decision}  = [diamond, minimum width=2cm, minimum height=1cm, text centered, draw=black, fill=green!30]
\tikzstyle{image}     = [minimum width=1cm, minimum height=1cm, text centered]
\tikzstyle{arrow}     = [thick,->,>=stealth]


% Do not change ConfigExecutive/config.tex file unless you really know what you are doing. 
% This file ends the configuration procedures (e.g. customizing commands, definition of new commands)
\input{ConfigExecutive/config}

% Insert here the info that will be displayed into your Title page 
% -> title of your work
\renewcommand{\title}{Simulating Aeration at Birth: building an Open-Source Newborn Lung Model}
% -> author name and surname
\renewcommand{\author}{Luca Andriotto}
% -> MSc course
\newcommand{\course}{Biomedical Engineering - Ingegneria Biomedica}
% -> advisor name and surname
\newcommand{\advisor}{Prof. Prof. Raffaele Dellaca'}
% IF AND ONLY IF you need to modify the co-supervisors you also have to modify the file ConfigExecutive/title_page.tex (ONLY where it is marked)
\newcommand{\firstcoadvisor}{Dr. Chiara Veneroni} % insert if any otherwise comment
%\newcommand{\secondcoadvisor}{Name Surname} % insert if any otherwise comment
% -> academic year
\newcommand{\YEAR}{2023-2024}

%%% Local Variables:
%%% mode: LaTeX
%%% TeX-master: "../Executive"
%%% End:


%----------------------------------------------------------------------------
%	THESIS BEGINS HERE
%----------------------------------------------------------------------------

\begin{document}

\fancypagestyle{plain}{%
  \fancyhf{} % Clear all header and footer fields
  \fancyhead[RO,RE]{\thepage} %RO=right odd, RE=right even
  \renewcommand{\headrulewidth}{0pt}
  \renewcommand{\footrulewidth}{0pt}}

%----------------------------------------------------------------------------
%	TITLE PAGE
%----------------------------------------------------------------------------

\pagestyle{empty} % No page numbers
\frontmatter % Use roman page numbering style (i, ii, iii, iv...) for the preamble pages

\puttitle{
  title   = Title,                                           % Title of the thesis
  name    = Luca Andriotto,                                  % Author Name and Surname
  course  = Biomedical Engineering - Ingegneria Biomedica,   % Study Programme (in Italian)
  ID      = 928454,                                          % Student ID number (numero di matricola)
  advisor = Professor Raffaele Dellaca',                        % Supervisor name
  coadvisor={Prof.ssa Chiara Veneroni},                  % Co-Supervisor name, remove this line (and look for "coadvisors" in `polimi3i_thesis.cls`) if there is none
  academicyear={2023-24},                                    % Academic Year
} % These info will be put into your Title page 

%----------------------------------------------------------------------------
%	PREAMBLE PAGES: ABSTRACT (inglese e italiano), EXECUTIVE SUMMARY
%----------------------------------------------------------------------------
\startpreamble
\setcounter{page}{1} % Set page counter to 1

% ABSTRACT IN ENGLISH
\chapter*{Abstract} 
\label{ch:abstract}%
\input{abstract.tex}

% ABSTRACT IN ITALIAN
\chapter*{Abstract in lingua italiana}
\label{ch:abstract_it}%
Qui va l'Abstract in lingua italiana della tesi seguito dalla lista di parole chiave.
\\
\\
\textbf{Parole chiave:} qui, vanno, le parole chiave, della tesi % Keywords (italian)


%----------------------------------------------------------------------------
%	LIST OF CONTENTS/FIGURES/TABLES/SYMBOLS
%----------------------------------------------------------------------------

% TABLE OF CONTENTS
\thispagestyle{empty}
\tableofcontents % Table of contents 
\thispagestyle{empty}
\cleardoublepage

%-------------------------------------------------------------------------
%	THESIS MAIN TEXT
%-------------------------------------------------------------------------
% In the main text of your thesis you can write the chapters in two different ways:
%
%(1) As presented in this template you can write:
%    \chapter{Title of the chapter}
%    *body of the chapter*
%
%(2) You can write your chapter in a separated .tex file and then include it in the main file with the following command:
%    \chapter{Title of the chapter}
%    \input{chapter_file.tex}
%
% Especially for long thesis, we recommend you the second option.

\addtocontents{toc}{\vspace{2em}} % Add a gap in the Contents, for aesthetics
\mainmatter % Begin numeric (1,2,3...) page numbering

% --------------------------------------------------------------------------
% NUMBERED CHAPTERS % Regular chapters following
% --------------------------------------------------------------------------
\chapter*{Introduction}
\input{introduction.tex}

\chapter{Chapter one}
\label{ch:chapter_one}%
% The \label{...}% enables to remove the small indentation that is generated, always leave the % symbol.
\input{chapter1.tex}

\chapter{Conclusions and future developments}
\label{ch:conclusions}%
\input{conclusions.tex}

%-------------------------------------------------------------------------
%	BIBLIOGRAPHY
%-------------------------------------------------------------------------

\addtocontents{toc}{\vspace{2em}} % Add a gap in the Contents, for aesthetics
\printbibliography

%-------------------------------------------------------------------------
%	APPENDICES
%-------------------------------------------------------------------------

\cleardoublepage
\addtocontents{toc}{\vspace{2em}} % Add a gap in the Contents, for aesthetics
\appendix
\chapter{Appendix A}
\input{appendixA.tex}

\chapter{Appendix B}
It may be necessary to include another appendix to better organize the presentation of supplementary material.



% LIST OF FIGURES
\listoffigures

% LIST OF TABLES
\listoftables

% LIST OF SYMBOLS
% Write out the List of Symbols in this page
\chapter*{List of Symbols} % You have to include a chapter for your list of symbols (
\input{symbols.tex}

% ACKNOWLEDGEMENTS
\chapter*{Acknowledgements}
\section*{Acknowledgements}

Vorrei innanzitutto ringraziare il Professor Raffaele Dellaca' e la
Dott.ssa Chiara Veneroni per l'opportunità di lavorare in laboratorio
e per il loro grande aiuto in questi mesi, dalla fase di progettazione
a quella di scrittura.  Ringrazio gli studenti e i dottorandi di
«TechRes» per i tutti i consigli e i bei momenti trascorsi insieme.

Un ringraziamento speciale va alla Dott.ssa Francesca Pennati, al
Dott. Cavigioli, e al Dott. Campari per il loro contributo nella fase
di Image Processing.

Un grazie particolare alla mia famiglia, senza la quale non avrei
avuto il privilegio di frequentare l'Università che desideravo.  In
particolare, dedico un pensiero a mio nonno che oggi non può essere
presente ma che potrà leggere il manoscritto in seguito.  Vi porto
sempre nel cuore.

Ringrazio di cuore Noemi che mi ha incoraggiato nei momenti di
difficoltà.

Ringrazio calorosamente gli amici, di una vita e nuovi.

Un grazie anche a tutti coloro che hanno anche solo minimamente
creduto in me.

Grazie a tutti, non sarebbe stato lo stesso senza ciascuno di voi.

%%% Local Variables:
%%% mode: LaTeX
%%% TeX-master: "../Thesis"
%%% End:


\cleardoublepage

\end{document}

%----------------------------------------------------------------------------
%	THESIS ENDS HERE
%----------------------------------------------------------------------------
