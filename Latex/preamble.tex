% A LaTeX template for ARTICLE version of the MSc Thesis submissions to 
% Politecnico di Milano (PoliMi) - School of Industrial and Information Engineering
%
% S. Bonetti, A. Gruttadauria, G. Mescolini, A. Zingaro
% e-mail: template-tesi-ingind@polimi.it
%
% Last Revision: October 2021
%
% Copyright 2021 Politecnico di Milano, Italy. Inc. NC-BY

\documentclass[11pt,a4paper]{article} 

% ------------------------------------------------------------------------------
%	REQUIRED PACKAGES AND  CONFIGURATIONS
%------------------------------------------------------------------------------
% PACKAGES FOR TITLES
\usepackage{titlesec}
\usepackage{color}

% PACKAGES FOR LANGUAGE AND FONT
\usepackage[utf8]{inputenc}
\usepackage[english]{babel}
\usepackage[T1]{fontenc} % Font encoding

% PACKAGES FOR IMAGES
\usepackage{graphicx}
\graphicspath{{Images/}}
\usepackage{eso-pic} % For the background picture on the title page
\usepackage{subfig} % Numbered and caption subfigures using \subfloat
\usepackage{caption} % Coloured captions
\usepackage{transparent}

% STANDARD MATH PACKAGES
\usepackage{amsmath}
\usepackage{amsthm}
\usepackage{bm}
\usepackage[overload]{empheq}  % For braced-style systems of equations

% PACKAGES FOR TABLES
\usepackage{tabularx}
\usepackage{longtable} % tables that can span several pages
\usepackage{colortbl}

% PACKAGES FOR ALGORITHMS (PSEUDO-CODE)
\usepackage{algorithm}
\usepackage{algorithmic}

% PACKAGES FOR REFERENCES & BIBLIOGRAPHY
\usepackage[colorlinks=true,linkcolor=black,anchorcolor=black,citecolor=black,filecolor=black,menucolor=black,runcolor=black,urlcolor=black]{hyperref} % Adds clickable links at references
\usepackage{cleveref}
\usepackage[backend=biber,style=numeric]{biblatex}
\usepackage{csquotes}
\addbibresource{Thesis_bibliography.bib}

% PACKAGES FOR THE APPENDIX
\usepackage{appendix}

% PACKAGES FOR ITEMIZE & ENUMERATES 
\usepackage{enumitem}

% OTHER PACKAGES
\usepackage{amsthm,thmtools,xcolor} % Coloured "Theorem"
\usepackage{comment} % Comment part of code
\usepackage{fancyhdr} % Fancy headers and footers
\usepackage{lipsum} % Insert dummy text
\usepackage{tcolorbox} % Create coloured boxes (e.g. the one for the key-words)

%-------------------------------------------------------------------------
%	NEW COMMANDS DEFINED
%-------------------------------------------------------------------------
% EXAMPLES OF NEW COMMANDS -> here you see how to define new commands
\newcommand{\bea}{\begin{eqnarray}} % Shortcut for equation arrays
\newcommand{\eea}{\end{eqnarray}}
\newcommand{\e}[1]{\times 10^{#1}}  % Powers of 10 notation
\newcommand{\mathbbm}[1]{\text{\usefont{U}{bbm}{m}{n}#1}} % From mathbbm.sty
\newcommand{\pdev}[2]{\frac{\partial#1}{\partial#2}}
% NB: you can also override some existing commands with the keyword \renewcommand

%----------------------------------------------------------------------------
%	ADD YOUR PACKAGES (be careful of package interaction)
%----------------------------------------------------------------------------
\usepackage{siunitx}
\usepackage[siunitx]{circuitikz}
\usepackage{tikz}
\ctikzset{bipoles/length=1.0cm}

%----------------------------------------------------------------------------
%	ADD YOUR DEFINITIONS AND COMMANDS (be careful of existing commands)
%----------------------------------------------------------------------------


% Do not change Configuration_Files/config.tex file unless you really know what you are doing. 
% This file ends the configuration procedures (e.g. customizing commands, definition of new commands)
\input{Configuration_Files/config}

% Insert here the info that will be displayed into your Title page 
% -> title of your work
\renewcommand{\title}{<Title of my Thesis>}
% -> author name and surname
\renewcommand{\author}{Luca Andriotto}
% -> MSc course
\newcommand{\course}{Biomedical Engineering - Ingegneria Biomedica}
% -> advisor name and surname
\newcommand{\advisor}{Prof. Raffaele Dellacà}
% IF AND ONLY IF you need to modify the co-supervisors you also have to modify the file Configuration_Files/title_page.tex (ONLY where it is marked)
\newcommand{\firstcoadvisor}{Prof.ssa Chiara Veneroni} % insert if any otherwise comment
\newcommand{\secondcoadvisor}{} % insert if any otherwise comment
% -> author ID
\newcommand{\ID}{928454}
% -> academic year
\newcommand{\YEAR}{2023-2024}
% -> abstract (only in English)
% Riassunto prima parte introduzione.
% Riscrivo l'aim così com'è.
% Riassunto prima parte conclusioni.

\renewcommand{\abstract}{
% Riassunto prima parte introduzione.
During pregnancy, the fetal airways are filled with a fluid known as
fetal lung fluid, which is essential for the development of airway
width. Consequently, at birth, the respiratory system must expel this
fluid to allow aeration. Fluid reabsorption begins a few days before
birth. During natural childbirth, fluid is expelled from the mouth and
nose due to the compression of the neonate's chest. In full-term
infants, the likelihood of complications during aeration is very
low. However, the scenario is vastly different for preterm infants.
% Riscrivo l'aim così com'è.
Aims of this projects are:
\begin{description}
\item Generate a model from neonatal CT scans to optimize the
  generation of airways, ensuring they adhere to the morphometric
  characteristics at various ages.
\item Develop an open-source mechanical model that allows for the
  simulation of mechanical properties along with fluid dynamics.
\end{description}
% Riassunto prima parte conclusioni.
Using a CT scan of a newborn infant, we extracted the centreline of
major airways and the lobe surfaces.  We then reconstructed the
anatomy of the missing airways using a statistical algorithm
originally proposed for adult lungs, which we adapted for the newborn
lung. This algorithm assigned airway diameters based on proportions
measured in the newborn lung.  We implemented a mechanical analog of
the airway and acini in Julia. This model accounts for changes related
to aeration at birth, allowing the simulation of the flow of fetal
fluid towards the periphery as air enters the airways.
}

% \renewcommand{\abstract}{During pregnancy, the fetal airways are
% filled with a fluid known as fetal lung fluid, which is essential for
% the development of airway width. Consequently, at birth, the
% respiratory system must expel this fluid to allow air to enter and
% exit, a process necessary for breathing (aeration). Fluid reabsorption
% begins a few days before birth through chemical processes involving
% sodium channels, and during natural childbirth, fluid is expelled from
% the mouth and nose due to the compression of the neonate's chest. In
% full-term infants, the likelihood of complications during aeration is
% very low. However, the scenario is vastly different for preterm
% infants, who are born before 37 weeks of gestation compared to the
% typical 40 weeks of a normal pregnancy.  Although recruitment
% maneuvers have gained more interest in preterm ventilation, there is
% still no common medical strategy. Experimental procedures are tested
% on animals, presenting challenges in obtaining results due to the
% invasiveness of the procedures and associated ethical issues. In
% silico modeling of the adult lung has been useful for understanding
% pathophysiology and making diagnoses. Thus, the same approach could
% help analyze various recruitment strategies and their impact on the
% lung during initial aeration at birth. However, in silico models of
% neonatal lungs are limited to describing up to the first generation of
% the bronchial tree and are therefore inadequate for simulating the
% physiological changes that occur at birth.}

% -> key-words (only in English)
\newcommand{\keywords}{morphometric model, aeration process,
lung, newborn, respiratory system}


