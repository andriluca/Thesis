% Riassunto prima parte introduzione.
% Riscrivo l'aim così com'è.
% Riassunto prima parte conclusioni.

\renewcommand{\abstract}{
% Riassunto prima parte introduzione.
During pregnancy, the fetal airways are filled with a fluid known as
fetal lung fluid, which is essential for the development of airway
width. Consequently, at birth, the respiratory system must expel this
fluid to allow aeration. Fluid reabsorption begins a few days before
birth. During natural childbirth, fluid is expelled from the mouth and
nose due to the compression of the neonate's chest. In full-term
infants, the likelihood of complications during aeration is very
low. However, the scenario is vastly different for preterm infants.
% Riscrivo l'aim così com'è.
Aims of this projects are:
\begin{description}
\item Generate a model from neonatal CT scans to optimize the
  generation of airways, ensuring they adhere to the morphometric
  characteristics at various ages.
\item Develop an open-source mechanical model that allows for the
  simulation of mechanical properties along with fluid dynamics.
\end{description}
% Riassunto prima parte conclusioni.
Using a CT scan of a newborn infant, we extracted the centreline of
major airways and the lobe surfaces.  We then reconstructed the
anatomy of the missing airways using a statistical algorithm
originally proposed for adult lungs, which we adapted for the newborn
lung. This algorithm assigned airway diameters based on proportions
measured in the newborn lung.  We implemented a mechanical analog of
the airway and acini in Julia. This model accounts for changes related
to aeration at birth, allowing the simulation of the flow of fetal
fluid towards the periphery as air enters the airways.
}

% \renewcommand{\abstract}{During pregnancy, the fetal airways are
% filled with a fluid known as fetal lung fluid, which is essential for
% the development of airway width. Consequently, at birth, the
% respiratory system must expel this fluid to allow air to enter and
% exit, a process necessary for breathing (aeration). Fluid reabsorption
% begins a few days before birth through chemical processes involving
% sodium channels, and during natural childbirth, fluid is expelled from
% the mouth and nose due to the compression of the neonate's chest. In
% full-term infants, the likelihood of complications during aeration is
% very low. However, the scenario is vastly different for preterm
% infants, who are born before 37 weeks of gestation compared to the
% typical 40 weeks of a normal pregnancy.  Although recruitment
% maneuvers have gained more interest in preterm ventilation, there is
% still no common medical strategy. Experimental procedures are tested
% on animals, presenting challenges in obtaining results due to the
% invasiveness of the procedures and associated ethical issues. In
% silico modeling of the adult lung has been useful for understanding
% pathophysiology and making diagnoses. Thus, the same approach could
% help analyze various recruitment strategies and their impact on the
% lung during initial aeration at birth. However, in silico models of
% neonatal lungs are limited to describing up to the first generation of
% the bronchial tree and are therefore inadequate for simulating the
% physiological changes that occur at birth.}

% -> key-words (only in English)
\newcommand{\keywords}{morphometric model, aeration process,
lung, newborn, respiratory system}
