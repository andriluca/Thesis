\section*{Sommario}
Durante la gravidanza, le vie aeree fetali sono riempite con un
liquido noto come fluido polmonare fetale, essenziale per lo sviluppo
delle vie aeree. Di conseguenza, alla nascita, il sistema respiratorio
deve espellere questo liquido per permettere l'aerazione.  I neonati
pretermine potrebbero non essere in grado di raggiungere in modo
autonomo l’aerazione polmonare alla nascita. L'applicazione di una
forma d'onda di pressione positiva all'apertura delle vie aeree può
supportarli. Tuttavia, la migliore strategia di pressione per
promuovere l’aerazione polmonare senza danneggiare il fragile polmone
è ancora sconosciuta. La simulazione del modello può aiutare nella
definizione di tale strategia.
Gli obiettivi di questo progetto sono: 1) Generare un modello a
partire dalle scansioni TC neonatali per ottimizzare la generazione
delle vie aeree, assicurando che aderiscano alle caratteristiche
morfometriche a diverse età.  2) Sviluppare un modello meccanico
open-source che consenta la simulazione delle proprietà meccaniche
insieme alla dinamica dei fluidi.
Utilizzando una scansione TC di un neonato, abbiamo estratto la linea
centrale delle principali vie aeree e le superfici dei
lobi. Successivamente, abbiamo ricostruito l'anatomia delle vie aeree
mancanti utilizzando un algoritmo statistico originariamente proposto
per i polmoni degli adulti, che abbiamo adattato per il polmone del
neonato. Questo algoritmo ha assegnato i diametri delle vie aeree in
base alle proporzioni misurate nel polmone del neonato. Abbiamo
implementato un analogo meccanico delle vie aeree e degli acini in
Julia. Questo modello tiene conto dei cambiamenti relativi
all'aerazione alla nascita, permettendo la simulazione del flusso del
liquido fetale verso la periferia mentre l'aria entra nelle vie aeree.

% Durante la gravidanza, le vie aeree del feto sono riempite di un
% liquido noto come liquido polmonare fetale, essenziale per lo sviluppo
% della larghezza delle vie aeree. Di conseguenza, alla nascita, il
% sistema respiratorio deve espellere questo liquido per permettere
% l'entrata e l'uscita dell'aria, un processo necessario per la
% respirazione (aerazione). Il riassorbimento del liquido inizia qualche
% giorno prima del parto attraverso processi chimici che coinvolgono i
% canali del sodio e, durante il parto naturale, il liquido viene
% espulso dalla bocca e dal naso grazie alla compressione del torace del
% neonato. Nei neonati a termine, la probabilità di complicazioni
% durante l'aerazione è molto bassa. Tuttavia, lo scenario è molto
% diverso per i neonati pretermine, che nascono prima delle 37 settimane
% di gestazione rispetto alle tipiche 40 settimane di una gravidanza
% normale.  Sebbene le manovre di reclutamento abbiano suscitato un
% maggiore interesse nella ventilazione dei pretermine, non esiste
% ancora una strategia medica comune. Le procedure sperimentali vengono
% testate sugli animali, presentando sfide nel ottenere risultati a
% causa dell'invasività delle procedure e dei problemi etici
% associati. La modellizzazione in silico del polmone adulto è stata
% utile per comprendere la patofisiologia e fare diagnosi. Pertanto, lo
% stesso approccio potrebbe aiutare ad analizzare le diverse strategie
% di reclutamento e il loro impatto sul polmone durante l'aerazione
% iniziale alla nascita. Tuttavia, i modelli in silico dei polmoni
% neonatali sono limitati alla descrizione fino alla prima generazione
% dell'albero bronchiale e, pertanto, non sono adeguati a simulare i
% cambiamenti fisiologici che avvengono alla nascita.  \vspace{15pt}
% Vedi sommario di Mani20.

\begin{tcolorbox}[arc=0pt, boxrule=0pt, colback=bluePoli!60, width=\textwidth, colupper=white]
  \textbf{Parole chiave:} modello morfometrico, processo di aerazione, polmone, neonato, sistema respiratorio, proprietà meccaniche
\end{tcolorbox}
