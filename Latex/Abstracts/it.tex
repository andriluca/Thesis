\section*{Abstract in lingua italiana}
Durante la gravidanza, le vie aeree del feto sono riempite di un
liquido noto come liquido polmonare fetale, essenziale per lo sviluppo
della larghezza delle vie aeree. Di conseguenza, alla nascita, il
sistema respiratorio deve espellere questo liquido per permettere
l'entrata e l'uscita dell'aria, un processo necessario per la
respirazione (aerazione). Il riassorbimento del liquido inizia qualche
giorno prima del parto attraverso processi chimici che coinvolgono i
canali del sodio e, durante il parto naturale, il liquido viene
espulso dalla bocca e dal naso grazie alla compressione del torace del
neonato. Nei neonati a termine, la probabilità di complicazioni
durante l'aerazione è molto bassa. Tuttavia, lo scenario è molto
diverso per i neonati pretermine, che nascono prima delle 37 settimane
di gestazione rispetto alle tipiche 40 settimane di una gravidanza
normale.  Sebbene le manovre di reclutamento abbiano suscitato un
maggiore interesse nella ventilazione dei pretermine, non esiste
ancora una strategia medica comune. Le procedure sperimentali vengono
testate sugli animali, presentando sfide nel ottenere risultati a
causa dell'invasività delle procedure e dei problemi etici
associati. La modellizzazione in silico del polmone adulto è stata
utile per comprendere la patofisiologia e fare diagnosi. Pertanto, lo
stesso approccio potrebbe aiutare ad analizzare le diverse strategie
di reclutamento e il loro impatto sul polmone durante l'aerazione
iniziale alla nascita. Tuttavia, i modelli in silico dei polmoni
neonatali sono limitati alla descrizione fino alla prima generazione
dell'albero bronchiale e, pertanto, non sono adeguati a simulare i
cambiamenti fisiologici che avvengono alla nascita.  \vspace{15pt}

% Vedi sommario di Mani20.

\begin{tcolorbox}[arc=0pt, boxrule=0pt, colback=bluePoli!60, width=\textwidth, colupper=white]
  \textbf{Parole chiave:} modello morfometrico, processo di aerazione, polmone, neonato, sistema respiratorio
\end{tcolorbox}
