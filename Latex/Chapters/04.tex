% 4. Mechanical model of airways and acini
\section{Mechanical model of airways and acini}

% (cambiare `alveoli`) (paper in chat Lutchen). (no zsoft e
% cartilagine)

% Scrivere in breve la struttura circuitale che poi implemento
% (copia).

% L'equivalente elettrico delle proprietà dei tessuti è stato modificato
% per poter testare l'impatto delle modifiche fisiologiche durante il
% processo di aereazione.

The electrical equivalent of tissues properties is modified to test
the physiological change impact on aeration process.

% Inoltre, il liquido fetale è incomprimibile, l'aria invece non lo è,
% comportando la presenza di un elemento aggiuntivo nell'equivalente
% elettrico dell'aria, in quanto la compressibilità del gas è
% modellizzata attraverso una capacità. In aggiunta, bisogna tener conto
% del collasso delle vie aeree, perché durante il processo di aerazione
% si crea un'interfaccia aria-liquido, che a sua volta dà origine a una
% tensione superficiale, la quale deve essere superata per permettere lo
% spostamento del fluido nell'albero bronchiale.  Una volta superata, i
% diametri si possono ingrandire facendo aumentare il volume polmonare.

Moreover, fetal fluid is incompressible, whereas air is not. This
introduces an additional element in the electrical equivalent of air,
as the compressibility of the gas is modeled through a
capacitance. Additionally, it is necessary to consider the collapse of
the airways. During the aeration process, an air-liquid interface is
created, which in turn generates surface tension that must be overcome
to allow fluid movement within the bronchial tree. Once this surface
tension is overcome, the diameters can expand, leading to an increase
in lung volume.

\textcite{lutchen1997} have developed a mechanical lung model in
frequency-domain, describing airways and alveoli modules as displayed
in \cref{fig:airway_impedance} and \cref{fig:alveolus_impedance},
respectively.

% Circuiti copiato da lutchen1997. 
\begin{figure}[H]\centering
  \begin{circuitikz}[scale=1]
    % Circuit
    %% Main branch
    %%% Input Node
    \draw (1.5,0)
    node[ocirc] (circuit_IN) {}
    to[short, ] ++(1,0) coordinate(R_IN)
    ;
    
    %% Main branch
    \draw (R_IN)
    to[short] ++(.5,0)
    to[resistor, l=$R(n) / 2$] ++(1,0)
    to[inductor, l=$I(n) / 2$] ++(3,0) coordinate (C_g_IN)
    to[short] ++(1.5,0) coordinate (SW_IN)
    to[resistor, l=$R(n) / 2$] ++(3,0)
    to[inductor, l=$I(n) / 2$] ++(1,0)
    to[short, -*] ++(1,0) coordinate(circuit_OUT)
    ;

    %%% C_g branch
    \draw (C_g_IN)
    to[polar capacitor, invert, l_=$C_{\text{g}}(n)$, *-] ++(0,-3.5)
    node[ground]{} ++(0,0)
    ;

    %%% Zw branch
    \draw (SW_IN)
    to[twoport, t={$Z_{\text{w}}(n)$},bipoles/twoport/height=1.2, *-] ++(0, -3.5)
    node[ground]{} ++(0,0)
    ;

    \draw (circuit_OUT)
    -- ++(0, .5)
    to ++(1.88,0)
    node[draw, anchor=west, thick] {$Z(n-1)$};
    
    \draw (circuit_OUT)
    -- ++(0, -.5)
    to ++(1,0)
    node[draw, anchor=west, thick] {$Z(n-1-\Delta$)};

    % Nodes
    %% IN
    \draw[->, >=stealth, thick] (0,0) -- (circuit_IN) node[midway, above] {Z(n)};
  \end{circuitikz}
  \caption{Impedance ($Z$) of a given order ($n$) of a single airway
    generation is calculated via an acoustic transmission line
    analysis, which accounts for shunting into gas compression in the
    tube ($C_{\text{g}}(n)$) and into nonrigid airway walls
    ($Z_{\text{w}}$).  R: resistance; $\Delta$: recursion
    index\cite{lutchen1997}.}
  \label{fig:airway_impedance}

\end{figure}

%%% Local Variables:
%%% mode: LaTeX
%%% TeX-master: "../Thesis"
%%% End:

\input{Images/equivalent_alveolus_impedance.tex}

\textcite{mani2020} has defined a mechanical model in time-domain starting
from the modules here described and properly changed.

A first implementation has been performed on «CADENCE» platform.  This
has the advantage of parallelism and speed.  There are also some
drawbacks to this approach: this framework is designed to simulate
standard electrical components and it is not well-suited to develop
time- and current integral-dependent components.  Furthermore license
is proprietary and machine specific.  This limits the accessibility of
of model design process.

% Posso dire che l'alveolo è stato modificato (vedi Veneroni2024) a
% partire dall'alveolo mostrato da lutchen1997 perché non risolvibile
% nel dominio del tempo.

%%% Local Variables:
%%% mode: LaTeX
%%% TeX-master: "../Thesis"
%%% End:
