% 5. Model Development
%   5.1. Airway Tree
%     5.1.1. CT Image Processing: Lung Segmentation, Centerline and Radii Extraction
%     5.1.2. Generation of the Statistical Portion
%   5.2. Mechanical Simulator
%     5.2.1. Blocks Description
%     5.2.2. Callbacks' Role in State Variables Discontinuity Handling
%     5.2.3. Model Testing on A Subtree
\section{Model Development}
% Mettere enfasi sulla ricerca dei metodi perché Julia e non altro.

% Visione d'insieme (diagramma con flusso dell'informazione, data
% pipeline). Questo diagramma mostra a partire dagli input che ho
% avuto (TC) tutto l'iter del dato.

% Qui ha detto poi Chiara che conviene inserire delle informazioni
% riguardanti i modelli matematici utilizzati.

\Cref{fig:data_pipeline} provides a high-level overview of the main
operations performed.

% Each of the two classes of methods corresponds to one of the two
% models:

% \begin{enumerate}
% \item \emph{Chaste} library is utilized in the generation of
%   \emph{morphometric model}.
% \item \emph{Julia Programming Language} is employed to describe and
%   instantiate the \emph{mechanical model}, as well as to perform
%   \emph{simulations}.
% \end{enumerate}

% todo: modificare l'immagine sopra ad "anatomical surrogate generation" con uno screen di ParaView.
% todo: aggiungere immagini sopra al branch del modello meccanico.

\begin{figure}[H]\centering
  \begin{tikzpicture}[node distance=3cm]
    % Nodes
    %% Chaste & Morphometric Model
    \node (ct_acquisition)           [startstop] {CT\\Acquisition};
    % \node (img_ct_acquisition)       [image, above of=ct_acquisition, yshift=-1.0cm]                   {\includegraphics[width=2.0cm]{ct_acquisition.png}};
    %%% Upper branch
    \node (lobes_segmentation)       [process, right of=ct_acquisition, xshift=3.25cm, yshift=+2cm]    {Lobes\\segmentation};
    %%% Lower branch
    % \node (img_lobes_segmentation)   [image, above of=lobes_segmentation, xshift=-.1cm, yshift=-1.2cm] {\includegraphics[width=3.5cm]{lobes_segmentation.png}};
    \node (airways_segmentation)     [process, below of=lobes_segmentation, xshift=-2cm, yshift=-1cm]  {Airways\\segmentation};
    % \node (img_airways_segmentation) [image, above of=airways_segmentation, xshift=2cm, yshift=-1cm]   {\includegraphics[height=2.5cm]{airways_segmentation.png}};
    \node (centerline_extraction)    [process, below of=lobes_segmentation, xshift=+2cm, yshift=-1cm]  {Centerline\\extraction};
    \node (surrogate_generation)     [process, right of=ct_acquisition, xshift=9.5cm]                  {Anatomical surrogate\\generation};
    % \node (img_surrogate_generation) [image, above of=surrogate_generation, yshift=-.5cm]              {\includegraphics[width=3.5cm]{lobes_segmentation.jpg}};
    %% Julia & Mechanical Model
    \node (parameters_update)        [process, below of=ct_acquisition, yshift=-2cm]                   {Lungs parameters\\update};
    \node (model_instantiation)      [process, right of=parameters_update, xshift=1cm]                 {Mechanical model\\instantiation};
    \node (DAE_solution)             [process, right of=model_instantiation, xshift=1cm]               {DAE System\\solution};
    \node (simulations)              [startstop, right of=DAE_solution, xshift=+1cm]                   {Mechanical Simulations};

    % Arrows
    \draw [arrow] (ct_acquisition.east)                              -- ++(.5,0) coordinate(tmp)     |- (airways_segmentation.west);
    \draw [arrow] (airways_segmentation)                             -- (centerline_extraction);
    \draw [arrow] (centerline_extraction.east)                       -- ++(.5,0) coordinate(tmp1)    |- (surrogate_generation.west);
    \draw [arrow] (model_instantiation)                              -- (DAE_solution);
    \draw [arrow] (DAE_solution)                                     -- (simulations);
    \draw [arrow] (tmp)                                              |-   (lobes_segmentation.west);
    \draw [arrow] (lobes_segmentation)                               --   (tmp1|-lobes_segmentation) |- (surrogate_generation.west);
    \draw [dashed, thick, ->,>=stealth] (surrogate_generation.south) -- ++(0,-2.5)                   -| (parameters_update.north);
    \draw [arrow] (parameters_update)                                -- (model_instantiation);

    % Frame around a part of the flowchart
    \begin{pgfonlayer}{background}
      \node[rounded corners=3mm,
      draw=blue,
      thick, dashed,
      % fit=(airways_segmentation)(centerline_extraction)(img_lobes_segmentation)(img_surrogate_generation),
      fit=(airways_segmentation)(centerline_extraction)(lobes_segmentation)(surrogate_generation),
      fill=cyan!5,
      inner sep=7pt,
      label={[anchor=south]above:\textsc{\textcolor{blue}{Morphometric model}}}] {};
    \end{pgfonlayer}
    \begin{pgfonlayer}{background}
      \node[rounded corners=3mm,
      draw=red,
      thick, dashed,
      fit=(parameters_update)(model_instantiation)(DAE_solution),
      fill=magenta!5,
      inner sep=7pt,
      label={[anchor=south]above:\textsc{\textcolor{red}{Mechanical model}}}] {};
    \end{pgfonlayer}
  \end{tikzpicture}
  \caption{Data pipeline.  The process begins with a
    \emph{patient-specific image} (i.e. CT) of a premature newborn.
    The extracted data, comprising \emph{two segmentations}, are then
    processed to obtain an anatomical surrogate of the airway tree.
    This is necessary due to scanner resolution not allowing for the
    discrimination and localization of small branches.  From the
    resulting morphometric model, the \emph{mechanical parameters} can
    be derived, which are essential for generating an accurate
    simulation model.  Finally, a numerical solver for differential
    equations provides the final output.}
    \label{fig:data_pipeline}
\end{figure}

%%% Local Variables:
%%% mode: LaTeX
%%% TeX-master: "../Thesis"
%%% End:


%   2.2. Morphometric model
\subsection{Airway Tree}
\label{subsec:airway_development}
% Per Chiara: Inserisco definizione di modello morfometrico?
There are different open-source platforms available for generating morphometric models.  In particular:

\begin{enumerate}
\item AVATree (Windows-only)
\item Chaste (crossplatform) library
\end{enumerate}

Due to problems related to «AVATree» source code compilation for
Windows with «VisualStudio», Chaste library is selected for this
project.

Chaste is a C++ open-source (BSD licensed) library developed by Oxford
University.  It has multiple use cases across various biomedical
fields, with an emphasis on cardiac electrophysiology and cancer
development\cite{mirams2013}.  It can be integrated into a C++ program
or used via «User Project» (i.e. \texttt{ctest}).  Specifically, the
``AirwayGenerationTutorial'' is considered as a first codebase and
properly adapted to match newborn
parameters\cite{airwaygeneration2024}.

The required input consists of two pieces of information:
\begin{itemize}
\item A \emph{mesh of centerline points}.  This mesh is provided in
  TetGen format, comprising ``airways.node'' and ``airways.edge''
  files. The first file lists centerline points coordinates,
  respective sampled airways radius and a boolean value to indicate if
  the point is generative. The second one contains all the connections
  between pairs of points.
\item Four (or five) \emph{lobes segmentations} in STL format.  These
  segmentations are necessary as they physically impose a limit on the
  growth algorithm.
\end{itemize}

%     2.2.1. CT, centerline and radii extraction
\subsubsection{CT Image Processing: Lung Segmentation, Centerline and
  Radii Extraction}
\label{subsubsec:ct_centerline_radii_extraction}
% todo: Devo chiedere bene a Francesca come sono stati estratti i
% raggi.

«3D Slicer» is an open-source software used for CT image processing.
Two extensions are installed:

\begin{description}
\item «\emph{Chest\_imaging\_platform}»: This extension enables
  semi-automatic segmentation of major airways from a single fiducial
  point.  It can also extract adult lobes using three fiducial points
  per lung fissure. However, in our case, the fissures are not
  visible, necessitating manual intervention.
\item «\emph{SlicerVMTK}»: This extension is used for extracting
  centerline points.
\end{description}

%     2.2.2. Generation of the statistical part
\subsubsection{Generation of the Statistical Portion}
\label{subsubsec:statistical_generation}

The Chaste User Project reads the input files
(see \Cref{subsec:airway_development}), and begins growing the
anatomical surrogate from the points labeled as «generative».  The
algorithm operating under the hood is a modified version of the one
described in \cite{tawhai2000,bordas2015}.  The generated output is
available in various formats:

\begin{itemize}
\item vtu: Unstructured Grid (base64 encoded) format used by VTK
  library.  It can be displayed by ParaView, an open-source viewer.
\item node and edge: TetGen format.  Such files are better suited for
  further processing.
\end{itemize}

% bordas2015 parla anche di come vengono generate le vie aeree e come
% vengono assegnati i diametri.

% Riferirsi a tawhai2000 e bordas2015.
This process is required as it is not possible to obtain high
generations (aka small airways) by means of standard high-resolution
CT\cite{bordas2015}.

The algorithm is based on a modified version of \textcite{tawhai2000}.
% {1} va spiegato qui, poi descriviamo l'algoritmo
A uniform grid of seed points is created within each segmented lobar
surface. Seed points approximately correspond to terminal bronchioles.
Spacing of the seed point grid is set so that the mean volume around
each of such points corresponded to the acinar volume (for adult being
$187\text{mm}$).

The starting points of the algorithm are the distal ends of the
segmented airway centerlines.  These points are referred to as growth
apices.

An \emph{adaptive threshold} on the distance between the seed points
and growth apices is required to prevent spurious long airways being
generated in the last few generations.  \Cref{eq:airway_threshold}
describes such threshold:

\begin{equation}
  T = \max(V_{\text{b}} - n\cdot D_{\text{l}}, 5\text{mm})
  \label{eq:airway_threshold}
\end{equation}

Where:
\begin{description}
\item $V_{\text{b}}$ is the diagonal size of the bounding box of the lobe being
  generated into
\item $D_{\text{l}} (= {V_{\text{b}}/{N}})$ is the distance limit.
\item $N$ is the maximum number of generations.
\item $n$ is the current generation number.
\end{description}

With these definitions, the \textbf{growing algorithm} is described as
such:

\begin{enumerate}
\item \emph{Each seed point is associated to the closest growth apex
    within its lobe}.  Seed point having a distance with respect to a
  growth apex greater than the aforementioned adaptive threshold is
  not associated to that distal end.  If all distal ends are further
  than the threshold from the seed point, the seed point remains
  unlabeled.
\item \emph{Calculation of the centroid of points assigned to each
    distal branch.}
\item \emph{The plane defined by the centroid and the parent branch is
    used to split the points into two unequal sets.}
\item \emph{Centroids of each of the new point sets are calculated.}
\item \emph{For each set of points a new airway is generated} starting
  at the distal end and extending 40\% of the distance towards the
  centroid of the point set.
\item \emph{Generated branches are checked to determine whether it is
    terminal}.  Branches whose length is less than $2\text{mm}$ are
  considered terminal.  Also branches whose point set contains just a
  single point solely are considered terminal points.  For all
  terminal branches their associated seed point is discarded from the
  global set.
\item \emph{Iterate} until no seed point is available.
\end{enumerate}

\textbf{Diameters} are computed by means of \Cref{eq:lumen_diameter}.

\begin{equation}
  \log D(x) = (x - N)\log(R_{\text{d}} H) + \log(D_{\text{N}})
  \label{eq:lumen_diameter}
\end{equation}

Where:
\begin{description}
\item $D$ is the aiway diameter.
\item $x$ is the current Horsfield order.
\item $N$ is the maximum Horsfield order.
\item $D_{\text{N}}$ is the maximum diameter.
\item $R_{\text{d}} H$ is the anti-log of the slope of airway diameter
  plotted against Horsfield order and is set to 1.15.  This parameter
  in the code is named ``DiameterRatio''.
\end{description}
  
%   2.3. Julia Programming Language
\subsection{Mechanical Simulator}
\label{subsec:simulator_development}

In order to perform simulation it is required to use an efficient
differential equation solver.  «DifferentialEquations.jl» wraps all
available solvers (even C and Fortran ones) and it is very efficient
\cite{diffeqdocs2024,rackauckas2017}.

Julia is a free, open-source (MIT licensed), fast, scientific and
numerical computing-oriented programming language.  Its computational
efficiency is comparable to that of statically-typed languages like C
or Fortran.  Moreover, its high-level code expressivity rivals that of
languages like Python, R and MATLAB\cite{juliadocs2024}.

Two key features, inspired by the \emph{Lisp Language}, are
highlighted here.

\begin{description}
\item \emph{Metaprogramming}: Code is treated as any other Julia data
  structure, thus can be dynamically generated and manipulated at
  runtime.
\item \emph{Macros}: They help instantiate the generated code in the
  body of a program.
\end{description}

Their importance is closely tied to the concept of Domain-Specific
Languages (aka DSLs).  These dialects are composed by abstractions
that can be properly exploited to solve particular problems
(e.g. modeling complex systems, solving differential equations).

Julia REPL has a built-in package manager (i.e. «\texttt{Pkg.jl}»)
used for managing project dependencies and ensuring the
\emph{repeatability} of computational setups.  This is achieved by
saving the required package names and commits into `Project.toml' and
`Manifest.toml' files.

%     2.3.1. «ModelingToolkit.jl» handles model complexity
\subsubsection{«\texttt{ModelingToolkit.jl}» handles
  Model Complexity}
\label{subsubsec:modelingtoolkit}

This Julia package encompasses all the tools necessary for model
design.  \texttt{ModelingToolkit.jl}» is equation-driven, requiring
each system to be described by Differential-Algebraic Equations
(i.e. DAEs) for subsequent solving\cite{ma2021}. Its built-in DSL
optimizes every stage of modeling, from prototyping components to
instantiating the complete system.

An acausal paradigm can be adopted, allowing users to reason in terms
of \emph{components}\cite{mtkdocs2024}. This modularity facilitates
system extensibility compared to the causal approach, where the entire
system of Differential-Algebraic Equations must be considered and
manually simplified\cite{ma2024}.

In particular, the usage of \jlinl{@mtkmodel} macro enables
hierarchical generation of building blocks recurring in the
highest-order model (i.e. «Lungs»).  Here is how information is
structured within \jlinl{@mtkmodel} macro.

\begin{jllisting}[label=@mtkmodel, caption={\jlinl{@mtkmodel}: a macro for systems prototyping.}]
  @mtkmodel <name_of_model> begin
      @parameters begin
          # (Optional) Some constant (e.g. Resistance, Capacitance) ...
      end
      @components begin
          # (Optional) Some dependency system (e.g. Resistor, Capacitor) ...
      end
      @variables begin
          # (Optional) Internal variables ...
      end
      @equations begin
          # Differential Algebraic Equations describing the model's behavior.
      end
      @continuous_events begin
          # (Optional) Some callback function ...
      end
  end
\end{jllisting}

Replicating the behavior of electrical components using this language
is straightforward, once you are familiar with the syntax and
understand the Differential-Algebraic Equations that represent their
characteristics.  Each generated system can then be composed into more
complex ones, using the internal \jlinl{@components} macro, thereby
implementing the hierarchical structure mentioned earlier.

After describing the highest-order system, Julia compiler requires its
instantiation before any simulation can be performed.  This is
accomplished using \jlinl{@mtkbuild} macro, which minimizes the number
of equations that need to be solved.

%     2.4.1. Blocks description
\subsubsection{Blocks Description}
\label{subsubsec:blocks_description}

Code modularity is directly reflected in the electrical equivalent
circuit.  Specifically, by encapsulating systems with the internal
\jlinl{@components} macro, it becomes possible to generate models of
increasing complexity.  This approach enables a clear separation
between components belonging to different hierarchical levels and
facilitate compartmentalization during the model design phase.

% 1. È necessario indicare la notazione utilizzata nelle formule?
% 2. Su cosa devo concentrarmi nella descrizione? Devo mostrare
% codice?

The following blocks are listed in a bottom-up order (from lowest to
highest).

\begin{enumerate}
\item \textbf{Electrical components}.  The simplest blocks are derived
  from «\texttt{ModelingToolkitStandardLibrary}», while
  integral-dependent ones rely on a modified mathematical block to
  manage both current integration and its timing correctly.  Their
  behavior varies based on the neonatal pulmonary fluid interface.
  \begin{itemize}
  \item \emph{Current Integral-Dependent Inductor}:
    $L(t) = L_{a} + L_{b}\cdot \left(1 - \dfrac{\int {i
          dt}}{V_{\text{FRC}}}\right)$
  \item \emph{Current Integral-Dependent Resistor}:
    $R(t) = R_{a} + R_{b}\cdot \left(1 - \dfrac{\int {i
          dt}}{V_{\text{FRC}}}\right)$
  \item \emph{Diode}
  \item \emph{Inductor}
  \item \emph{Resistor}
  \end{itemize}
\item \textbf{Modules}.  Obtained by connecting the aforementioned
  components together into functional models representing a
  physiological structure.
  \begin{itemize}
  \item
    \emph{Alveolus}
  \item \emph{Airway}.  It has a similar behavior with respect to a
    transmission line.
  \end{itemize}
\item \textbf{Lungs}.  Highest order model as it is a combination of
  alveoli and airways.
\end{enumerate}

% Equivalent circuits for airways and alveolus.
\begin{figure}[H]\centering
  \begin{circuitikz}[]
    % Circuit
    %% Main branch

    %%% Input Node
    \draw (1.5,0)
    node[ocirc] (circuit_IN) {}
    to[short, i=$i_{\text{in}}$, -*] ++(1.5,0) coordinate(D_SW_IN)
    ;
    
    %%% Diode // Switch branch
    \draw (D_SW_IN)
    to[short] ++(0,.5)
    to[diode, v^<=$V_{\text{in,th}}$, color=bluePoli] ++(2,0) coordinate(D_SW_OUT)
    to[short, -*] (D_SW_IN-|D_SW_OUT)
    ;
    \draw (D_SW_IN)
    to[short] ++(0,-.5)
    to[switch, color=bluePoli, a_=$\left(V_{\text{in}}\geq V_{\text{in,th}}\right) \lor \left(V_{\text{in}} \text{ = } 0\right)$] ++(2,0)
    to[short] (D_SW_IN-|D_SW_OUT)
    ;

    %% Main branch
    \draw (D_SW_IN-|D_SW_OUT)
    to[short] ++(.5,0)
    to[variable resistor, color=bluePoli, l=$R_{\text{tube}} / 2$] ++(2,0)
    to[variable inductor, color=bluePoli, l=$I_{\text{tube}} / 2$] ++(2,0) coordinate (C_g_IN)
    to[short] ++(1.5,0) coordinate (SW_IN)
    to[variable resistor, color=bluePoli, l=$R_{\text{tube}} / 2$] ++(2,0)
    to[variable inductor, color=bluePoli, l=$I_{\text{tube}} / 2$] ++(2,0)
    to[short, i=$i_{\text{out}}$] ++(1,0)
    node[ocirc] (circuit_OUT) {}
    ;

    %%% C_g branch
    \draw (C_g_IN)
    to[C=$C_{\text{g}}$, *-] ++(0,-3.5)
    node[ground]{} ++(0,0)
    ;

    %%% Sw branch
    \draw (SW_IN)
    to[L=$I_{\text{sw}}$, *-] ++(0,-1.5)
    to[R=$R_{\text{sw}}$] ++(0,-1)
    to[C=$C_{\text{sw}}$] ++(0,-1)
    node[ground]{} ++(0,0)
    ;
    
    %% Open circuit (input)
    \draw (circuit_IN)
    to[open, -o, v<=$V_{\text{in}}$] ++(0, -3.5)
    node[ground] {}
    ;
    %% Open circuit (output)
    \draw (circuit_OUT)
    to[open, -o, v^<=$V_{\text{out}}$] ++(0, -3.5)
    node[ground] {}
    ;

    % Nodes
    %% IN
    \draw  (circuit_IN.west)
    node[draw, anchor=east, color=white, fill=bluePoli!50] (node_IN) {IN}
    ;
    % \draw (node_IN.east) node[ocirc, right]{}
    % ;
    %% OUT
    \draw  (circuit_OUT.east)
    node[draw, anchor=west, color=white, fill=bluePoli!50] (node_OUT) {OUT}
    ;
  \end{circuitikz}
  \label{cir:airway}
  \caption{Airway equivalent circuit.  In blue: all current integral-dependent components.}

\end{figure}

\input{Images/equivalent_alveolus.tex}

%     2.3.2. Callbacks role in state variables discontinuity handling
\subsubsection{Callbacks' Role in State Variables Discontinuity
  Handling}
\label{subsubsec:callbacks}

Not all characteristics of electrical components can be defined solely
by DAEs.  Voltages or currents may suddently change, triggered by a
circuit event.  In such cases, \emph{continuous callback functions}
can be employed to appropriately alter the value of state variables.
These callbacks consist of two functions:
\begin{itemize}
\item \texttt{condition}: Specifies the event to be tested.
\item \texttt{affect}: Defines how the state variable(s) should be
  changed.
\end{itemize}

The component-based approach allows for the definition of callbacks
directly within the (sub)system being modeled.

% Example of usage?

%     2.4.2. Model testing on a subtree
\subsubsection{Model Testing on A Subtree}
\label{subsubsec:model_testing_on_subtree}

Simulations are executed starting from a subnet, as the full circuit
(comprising over 50k modules) requires more memory space than
typically available on a common laptop.

\begin{figure}[H]
  \centering
  \includegraphics[scale=.7]{subtree.pdf}
  \caption{The simulated subtree.  Airways are represented in light blue, alveoli in light green.}
  \label{fig:subtree}
\end{figure}

% In this project, Chaste has been relied upon generating an anatomical
% surrogate for lungs in premature newborns.

%%% Local Variables:
%%% mode: LaTeX
%%% TeX-master: "../Thesis"
%%% End:
