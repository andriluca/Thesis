\section{Model Development}
\label{sec:model_development}

% model development: copio la prima parte o la summarizzo, prendo
% quello che serve per spiegare le figure e poco più.

An anatomically coherent 3D lung model is combined with a mechanical
model of the airways and acini, able to simulate changes in the
mechanical properties of the airways when the lung fluid is replaced
by air entering the lungs.

The sequence for model development is reported in
\cref{fig:data_pipeline}. We extracted a 3D surface mesh of lung lobes
and airway centrelines from a lung CT of a newborn. We implemented a
statistical method, previously described for adult lung models, able
to generate distal airways that were not visible on the CT. We adapted
the method for the newborn lung.

We implemented a mechanical model of the airways and acini whose
parameters are dependent on the airway's lengths and diameters and the
presence of fetal fluid, fetal fluid-air interface, or air in the
airway. We exploited an open-source solver for differential equations
to simulate the network.

Morphology generation process is required as it is not possible to
obtain high generations (aka small airways) using standard
high-resolution CT\cite{bordas2015}.  This is performed by «Chaste» an
open-source C++ library for computational physiology.  The «Chaste»
User Project required for airways and lobar segmentations to calculate
the distal airways.

To perform simulation it is required to use an efficient differential
equation solver.  Julia Programming Language\cite{juliadocs2024} has
«\texttt{DifferentialEquations.jl}», which is very efficient and
includes all available solvers (even C and FORTRAN
ones)\cite{diffeqdocs2024,rackauckas2017}.
«\texttt{ModelingToolkit.jl}»\cite{ma2021} is also required: it is a
package for model design and instantiation.  It allows for prototyping
components easily, using macros and a Domain-Specific Language
optimized for such purpose.  Lung model is made out of modules at
various hierarchical levels: from the lowest ones at the level of the
electrical component to the highest ones emulating acini and airways
(see their schematics in \cref{fig:airway,fig:acinus}).

Simulations are executed starting from a subtree (see
\cref{fig:subtree_development}), as the full circuit (comprising over
50k modules) requires more memory space than typically available on a
common laptop.


%% ex Figure 3
\begin{figure*}[]
  \centering
  \scalebox{.75}{
    \begin{tikzpicture}[]
      % Nodes
      %% Chaste & Morphometric Model
      \node (ct_acquisition)           [startstop] {CT Acquisition};
      %%% Upper branch
      \node (lobes_segmentation)       [process, right of=ct_acquisition, xshift=8.25cm, yshift=+1cm]      {Lobes segmentation};
      \node (airways_segmentation)     [process, below of=lobes_segmentation, xshift=-2.5cm, yshift=-1cm]  {Airways segmentation};
      \node (centreline_extraction)    [process, below of=lobes_segmentation, xshift=+2.5cm, yshift=-1cm]  {Centreline extraction};
      \node (anatomical_generation)    [process, minimum width=5cm, text width=5cm, right of=ct_acquisition, xshift=16.5cm]                  {Algorithmic Generation of Distal Airway Centrelines};
      %% Julia & Mechanical Model
      \node (airways_model)            [startstop, below of=ct_acquisition, yshift=-5cm]               {Airways model};
      \node (acini_model)              [startstop, below of=airways_model, yshift=-1cm]                  {Acini model};
      \node (model_instantiation)      [process, below of=airways_segmentation, yshift=-5cm]                     {Mechanical model instantiation};
      \node (DAE_solution)             [process, below of=centreline_extraction, yshift=-5cm]               {DAE System solution};
      \node (tracheal_pressure)        [startstop, above of=DAE_solution, yshift=2.5cm]                      {Tracheal pressure waveform};
      \node (simulations)              [startstop, below of=anatomical_generation, yshift=-6cm]                   {Simulations results};
      \node (phi_length)               [startstop, minimum width=5cm, text width=4.5cm, above of=model_instantiation, yshift=2.5cm]                 {Diameters, length, father airway of each airway};
      
      
      % % Arrows
      \draw [arrow] (ct_acquisition.east)                              -- ++(1.5,0) coordinate(tmp)     |- (airways_segmentation.west);
      \draw [arrow] (airways_segmentation)                             -- (centreline_extraction);
      \draw [arrow] (centreline_extraction.east)                       -- ++(.5,0) coordinate(tmp1)    |- (anatomical_generation.west);
      \draw [arrow] (model_instantiation)                              -- (DAE_solution);
      \draw [arrow] (DAE_solution)                                     -- (simulations);
      \draw [arrow] (tmp)                                              |-   (lobes_segmentation.west);
      \draw [arrow] (lobes_segmentation)                               --   (tmp1|-lobes_segmentation) |- (anatomical_generation.west);
      \draw [arrow] (phi_length.south)  -- (model_instantiation.north);
      \draw [arrow] (acini_model.east) -- ++(1.5,0)  |- (model_instantiation.west);
      \draw [arrow] (tracheal_pressure.south) -- (DAE_solution.north);
      \draw [arrow] (airways_model.east) -- ++(1.5,0)  |- (model_instantiation.west);
      \draw [dashed, thick, ->,>=stealth] (12.5,-1.775) -- ++(0,-.50)                   -| (phi_length.north);

      
      % % Frame around a part of the flowchart
      \begin{pgfonlayer}{background}
        \node[rounded corners=3mm,
        draw=blue,
        thick, dashed,
        % fit=(airways_segmentation)(centreline_extraction)(img_lobes_segmentation)(img_anatomical_generation),
        fit=(airways_segmentation)(centreline_extraction)(lobes_segmentation)(anatomical_generation),
        fill=cyan!5,
        inner sep=7pt,
        label={[anchor=south]above:\textsc{\textcolor{blue}{Morphometric model}}}] {};
      \end{pgfonlayer}
      \begin{pgfonlayer}{background}
        \node[rounded corners=3mm,
        draw=red,
        thick, dashed,
        fit=(airways_model)(acini_model)(model_instantiation)(DAE_solution),
        fill=magenta!5,
        inner sep=7pt,
        label={[anchor=north]below:\textsc{\textcolor{red}{Mechanical model}}}] {};
      \end{pgfonlayer}
    \end{tikzpicture}
  }
  \caption{Data pipeline.  The process begins with a
    \emph{patient-specific image} (i.e. CT) of a premature newborn.
    The extracted data, comprising \emph{two segmentations}, are then
    processed to obtain an anatomical surrogate of the airway tree.
    This is necessary due to scanner resolution not allowing for the
    discrimination and localization of small branches.  From the
    resulting morphometric model, the \emph{mechanical parameters} can
    be derived, which are essential for generating an accurate
    simulation model.  Finally, a numerical solver for differential
    equations provides the final output.}
  \label{fig:data_pipeline}
\end{figure*}

%%% Local Variables:
%%% mode: LaTeX
%%% TeX-master: "../Executive"
%%% End:


%% ex Figure 4
\begin{figure*}[]\centering
  \begin{circuitikz}[scale=.9]
    % Circuit
    %% Main branch

    %%% Input Node
    \draw (1.5,0)
    node[ocirc] (circuit_IN) {}
    to[short, i=$i_{\text{in}}$, -*] ++(1.5,0) coordinate(D_SW_IN)
    ;
    
    %%% Diode // Switch branch
    \draw (D_SW_IN)
    to[short] ++(0,.5)
    to[diode, v^<=$V_{\text{in,th}}$, color=bluePoli] ++(2,0) coordinate(D_SW_OUT)
    to[short, -*] (D_SW_IN-|D_SW_OUT)
    ;
    \draw (D_SW_IN)
    to[short] ++(0,-.5)
    to[switch, color=bluePoli, a_=$\left(V_{\text{in}}\geq V_{\text{in,th}}\right) \lor \left(V_{\text{in}} \text{ = } 0\right)$] ++(2,0)
    to[short] (D_SW_IN-|D_SW_OUT)
    ;

    %% Main branch
    \draw (D_SW_IN-|D_SW_OUT)
    to[short] ++(.5,0)
    to[variable resistor, color=bluePoli, l=$R_{\text{tube}} / 2$] ++(2,0)
    to[variable inductor, color=bluePoli, l=$I_{\text{tube}} / 2$] ++(2,0) coordinate (C_g_IN)
    to[short] ++(1.5,0) coordinate (SW_IN)
    to[variable resistor, color=bluePoli, l=$R_{\text{tube}} / 2$] ++(2,0)
    to[variable inductor, color=bluePoli, l=$I_{\text{tube}} / 2$] ++(2,0)
    to[short, i=$i_{\text{out}}$] ++(1,0)
    node[ocirc] (circuit_OUT) {}
    ;

    %%% C_g branch
    \draw (C_g_IN)
    to[C=$C_{\text{g}}$, *-] ++(0,-3.5)
    node[ground]{} ++(0,0)
    ;

    %%% Sw branch
    \draw (SW_IN)
    to[L=$I_{\text{sw}}$, *-] ++(0,-1.5)
    to[R=$R_{\text{sw}}$] ++(0,-1)
    to[C=$C_{\text{sw}}$] ++(0,-1)
    node[ground]{} ++(0,0)
    ;
    
    %% Open circuit (input)
    \draw (circuit_IN)
    to[open, -o, v<=$V_{\text{in}}$] ++(0, -3.5)
    node[ground] {}
    ;
    %% Open circuit (output)
    \draw (circuit_OUT)
    to[open, -o, v^<=$V_{\text{out}}$] ++(0, -3.5)
    node[ground] {}
    ;

    % Nodes
    %% IN
    \draw  (circuit_IN.west)
    node[draw, anchor=east, color=white, fill=bluePoli!50] (node_IN) {IN}
    ;
    % \draw (node_IN.east) node[ocirc, right]{}
    % ;
    %% OUT
    \draw  (circuit_OUT.east)
    node[draw, anchor=west, color=white, fill=bluePoli!50] (node_OUT) {OUT}
    ;
  \end{circuitikz}
  \caption{Airway equivalent circuit.  In blue: all current integral-dependent components.}
  \label{fig:airway}

\end{figure*}

%%% Local Variables:
%%% mode: LaTeX
%%% TeX-master: "../Executive"
%%% End:


%% ex Figure 5
\begin{figure*}[]\centering
  \begin{circuitikz}[scale=.9]
    % Circuit
    %% Main branch

    %%% Input Node
    \draw (1.5,0)
    node[ocirc] (circuit_IN) {}
    to[short, i=$i_{\text{in}}$, -*] ++(1.5,0) coordinate(D_SW_IN)
    ;
    
    %%% Diode // Switch branch
    \draw (D_SW_IN)
    to[short] ++(0,.5)
    to[diode, v^<=$V_{\text{in,th}}$, color=bluePoli] ++(2,0) coordinate(D_SW_OUT)
    to[short, -*] (D_SW_IN-|D_SW_OUT)
    ;
    \draw (D_SW_IN)
    to[short] ++(0,-.5)
    to[switch, color=bluePoli, a_=$\left(V_{\text{in}}\geq V_{\text{in,th}}\right) \lor \left(V_{\text{in}} \text{ = } 0\right)$] ++(2,0)
    to[short] (D_SW_IN-|D_SW_OUT)
    ;

    %% Main branch
    \draw (D_SW_IN-|D_SW_OUT)
    % to[short] ++(.5,0)
    to[variable resistor, color=bluePoli, l=$R_{\text{tube}}$] ++(2.25,0)
    to[variable inductor, color=bluePoli, l=$I_{\text{tube}}$] ++(2.25,0) coordinate (circuit_OUT)
    % to[short] ++(.5,0)
    to[L=$I_{\text{t}}$] ++(2.25,0)
    to[R=$R_{\text{t}}$] ++(1.5,0)
    to[C=$C_{\text{t}}$] ++(1.5,0) coordinate (S_IN)
    % node[ocirc] (circuit_OUT) {}
    ;

    \draw (S_IN)
    to[short] ++(0,-.5)
    to[R, l_=$R_{\text{s}}$] ++(2,0)
    to[short] ++(0,.5)
    ;
    \draw (S_IN)
    to[short, *-] ++(0,.5)
    to[C=$C_{\text{s}}$] ++(2,0)
    to[short] ++(0,-.5)
    to[short, *-] ++(.5,0)
    node[ground]{} ++(0,0)
    ;

    %%% C_g branch
    \draw (circuit_OUT) node[draw, anchor=south, color=white, fill=bluePoli!50] {OUT}
    to[C=$C_{\text{g}}$, i=$i_{\text{out}}$, v<=$V_{\text{out}}$, *-] ++(0,-3.5)
    node[ground]{} ++(0,0)
    ;

    %% Open circuit (input)
    \draw (circuit_IN)
    to[open, -o, v<=$V_{\text{in}}$] ++(0, -3.5)
    node[ground] {}
    ;
    % %% Open circuit (output)
    % \draw (circuit_OUT)
    % {[red!60] to[open, -o, v^<=$V_{\text{out}}$] ++(0, -3.5)}
    % node[ground] {}
    % ;

    % Nodes
    %% IN
    \draw  (circuit_IN.west)
    node[draw, anchor=east, color=white, fill=bluePoli!50] (node_IN) {IN}
    ;
    % \draw (node_IN.east) node[ocirc, right]{}
    % ;
    %% OUT

  \end{circuitikz}
  \caption{Acinus equivalent circuit.  In blue: all current integral-dependent components.}
  \label{fig:acinus}

\end{figure*}

%%% Local Variables:
%%% mode: LaTeX
%%% TeX-master: "../Executive"
%%% End:


%% subtree
\begin{figure}[H]\centering
  \begin{tikzpicture}[node distance=1.3cm]
    \node (GEN) [rectangle, rounded corners, draw=black, , fill=green!25] {Generator};
    \node (IAD) [airway, below of=GEN]                                    {\textsc{IAD}};
    \node (IAF) [airway, below of=IAD]                                    {\textsc{IAF}};
    \node (IAH) [airway, below of=IAF]                                    {\textsc{IAH}};
    \node (IBL) [airway, below of=IAH]                                    {\textsc{IBL}};
    \node (IAE) [acinus, left of=IAD, xshift=-.25cm, yshift=-1.3cm]       {\textsc{IAE}};
    \node (IAG) [acinus, right of=IAF, xshift=.25cm, yshift=-1.3cm]       {\textsc{IAG}};
    \node (IAI) [acinus, left of=IAH, xshift=-.25cm, yshift=-1.3cm]       {\textsc{IAI}};
    \node (IBA) [acinus, right of=IBL, xshift=.25cm, yshift=-1.3cm]       {\textsc{IBA}};
    \node (IBB) [acinus, left of=IBL, xshift=-.25cm, yshift=-1.3cm]       {\textsc{IBB}};

    \draw [arrow1] (GEN) -- (IAD);
    \draw [arrow1] (IAD) -- (IAF);
    \draw [arrow1] (IAF) -- (IAH);
    \draw [arrow1] (IAH) -- (IBL);
    \draw [arrow1] (IAD) -- (IAE);
    \draw [arrow1] (IAF) -- (IAG);
    \draw [arrow1] (IAH) -- (IAI);
    \draw [arrow1] (IBL) -- (IBA);
    \draw [arrow1] (IBL) -- (IBB);
  \end{tikzpicture}
  \caption{The simulated subtree. Airways are represented in red,
    acini in yellow.}
  \label{fig:subtree_development}
\end{figure}


%%% Local Variables:
%%% mode: LaTeX
%%% TeX-master: "../Executive"
%%% End:
