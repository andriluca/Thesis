% 3. Anatomical Models
%   3.1. Existing Infant Lungs Models
\section{Anatomical Models}
% - quello di agnello (al-jumaily2011)
% - un altro (herrmann2016)

% I modelli matematici sviluppati per i polmoni degli adulti non possono
% essere semplicemente ridotti in scala per adattarsi ai polmoni dei
% neonati. Infatti, i polmoni dei neonati non sono semplicemente una
% versione in miniatura dei polmoni degli adulti, ma presentano
% differenze significative in termini di proporzioni dei rami
% bronchiali, costituenti delle vie aeree, caratteristiche morfometriche
% e composizione. Queste differenze devono essere prese in
% considerazione quando si sviluppano o si adattano modelli matematici
% per rappresentare accuratamente il funzionamento dei polmoni dei
% neonati. La struttura dei polmoni dei neonati presenta proporzioni
% diverse rispetto a quella degli adulti. Le diramazioni delle vie aeree
% possono avere dimensioni e disposizioni differenti.  I componenti dei
% polmoni, come il tessuto e le cellule, possono variare tra neonati e
% adulti.  Gli studi morfometrici, che analizzano la forma e la
% struttura dei polmoni, mostrano che ci sono differenze tra neonati e
% adulti che devono essere considerate nei modelli.  Ci sono differenze
% nella composizione del tessuto polmonare tra neonati e adulti che
% influenzano come i polmoni funzionano e rispondono alle terapie.

Mathematical models developed for adult lungs cannot simply be scaled
down to fit the lungs of newborns. In fact, newborn lungs are not
simply one miniature version of adult lungs, but they present
significant differences in terms of bronchial branch proportions,
constituents of the airways\cite{merkus1996}, morphometric
characteristics\cite{horsfield1987} and
composition\cite{hislop1989}. These differences must be taken into
account when developing or adapting mathematical models to accurately
represent the functioning of the lungs of the newborns. The structure
of the lungs of newborns presents proportions different than that of
adults. The branches of the airways they can have different sizes and
arrangements. The components of lungs, like tissue and cells, can vary
between newborns and adults. Morphometric studies, which analyze the
shape and the structure of the lungs, show that there are differences
between newborns and adults that need to be considered in the
models. There are differences in the composition of lung tissue
between newborns and adults who influence how the lungs function and
respond to therapies.

%% Citare qui i riferimenti che ha fatto chiara nella call del 21.06

Past work\cite{mani2020} considered an adult lung model linearly
scaled to match newborn anatomical features.  The advantage of this
approach is that it respects the dimensions of trachea and
bronchioles. It doesn't guarantee that the morphometric
characteristics of the entire airway tree are respected.  In this
work, there are few airway generation parameters that can in fact be
adapted, in order to better approximate the target morphometric
characteristics.

\subsection{Existing Infant Lungs Models}

% Modello ovino (al-jumaily2011) e di Jacob (herrmann2016).
In literature, there exist models based on the
ovine\cite{al-jumaily2011} and canine\cite{herrmann2016} anatomy.

%%% Local Variables:
%%% mode: LaTeX
%%% TeX-master: "../Thesis"
%%% End:
