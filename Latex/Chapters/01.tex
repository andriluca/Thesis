% 1. Introduction
\section{Introduction}
\label{sec:introduction}

During pregnancy, the fetal airways are filled with a fluid known as
fetal lung fluid, which is essential for the development of airway
width. Consequently, at birth, the respiratory system must expel this
fluid to allow air to enter and exit, a process necessary for
breathing (aeration). Fluid reabsorption begins a few days before
birth through chemical processes involving sodium channels, and during
natural childbirth, fluid is expelled from the mouth and nose due to
the compression of the neonate's chest. In full-term infants, the
likelihood of complications during aeration is very low. However, the
scenario is vastly different for preterm infants, who are born before
37 weeks of gestation compared to the typical 40 weeks of a normal
pregnancy.

Applying pressure at the entrance of the airways helps preterm infants
remove the fluid and open the closed acini. Once the acini are
recruited, less pressure is needed to keep them open and ventilate the
lung. Thus, employing recruitment strategies at birth has the
advantage of initiating ventilation in a more recruited lung, using
lower pressures, achieving more homogeneous lung aeration, and
reducing the stress applied to the tissues. It is important to note,
however, that at birth, the lung is more delicate and thus more
susceptible to damage, even with routine ventilation
procedures. Therefore, developing protective recruitment strategies at
birth could lead to significant improvements in this area.

Although recruitment maneuvers have gained more interest in preterm
ventilation, there is still no common medical strategy. Experimental
procedures are tested on animals, presenting challenges in obtaining
results due to the invasiveness of the procedures and associated
ethical issues\cite{al-jumaily2011, herrmann2016}. In silico modeling
of the adult lung has been useful for understanding pathophysiology
and making diagnoses. Thus, the same approach could help analyze
various recruitment strategies and their impact on the lung during
initial aeration at birth. However, in silico models of neonatal lungs
are limited to describing up to the first generation of the bronchial
tree resulting inadequate for simulating the physiological changes
that occur at birth.  Anatomical models of the adult lung were scaled
to match the newborns one. However, there are differences between
airway length and diameter proportion between infants and
adults\cite{horsfield1987}. Moreover, the immature lung structure
results in different mechanical properties\cite{merkus1996}.

% AGGIUNTA MODIFICA, COMPLETARE
Alternatively, anatomical properties .. scaled r.

Consequently, both the anatomical and mechanical characteristics of
the tissues must be appropriately modified to achieve a consistent
model for the neonatal case in the time domain.  Finally, to implement
the changes occurring at birth, the mechanical properties of each
airway must change when the fetal fluid is replaced by air.

A previous thesis work developed an in silico model able to simulate
the mechanical changes occurring in the airways during aeration at
birth in the time domain. However, the anatomical structure was scaled
for the adult one, and the mechanical model was implemented in
"CADENCE.", a platform optimised for the analysis of integrated
circuits. The use of «CADENCE» platform results in limitations related
to the need for proprietary software licenses and difficulties in
implementing time-varying phenomena not easily described by standard
electronic components.

Using open-source strategies for the model eliminates the need for
proprietary software licenses, making the development process more
accessible\cite{mani2020}.


Aims of this projects are:
\begin{description}
\item Generate a model from neonatal CT scans to optimize the
  generation of airways, ensuring they adhere to the morphometric
  characteristics at various ages.
\item Develop an open-source mechanical model that allows for the
  simulation of mechanical properties along with fluid dynamics
  \cite[][Ch. 1.9 - 1.11]{mani2020} (capillary pressure associated
  with the air-fluid interface).
\end{description}

%%% Local Variables:
%%% mode: LaTeX
%%% TeX-master: "../Thesis"
%%% End:
