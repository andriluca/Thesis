\section{Introduction}
\label{sec:introduction}
WIP.

% Commenti
% - Fragilità polmone neonatale
% - Differenze anatomiche tra neonato e adulto.
% - Albero bronchiale definizione etc. (penso anche ordini, branching etc).
% - Modello morfometrico definizione etc (?)
% - Modello polmone di agnello con capacità in parallelo.
% - Diametro trachea in funzione dell'eta gestazionale (grafico).

% \subsection{Premature newborns}
% \label{subsec:premature_newborns}

% Commenti
% Differenze tra sistema respiratorio nell'adulto e nel neonato e nel
% neonato prematuro.  (Slide 4 di [1]).

% \subsection{Morphometric Model}
% \label{subsec:morphometric_model}
% I am referring to \Cref{subsec:premature_newborns}

% \subsubsection{A brief definition}
% \label{subsubsec:morphometric_model_definition}

% Commenti
% Un'idea per questa definizione la si ritrova in
% `ModelloMorfometricoTriennale`.

% Commenti
% A morphometric model can be thought as a mathematical model composed
% by fixed parameters having both a physical and a geometrical (?)
% meaning.

% \subsubsection{Its structure}

% Slide 3 di [1]

% Riferimenti: `ModelloMorfometricoTriennale`,
% `ModelloMorfometricoTriennaleTesi`,
% `ModelloMorfometricoTriennaleBib/`

%%% Local Variables:
%%% mode: LaTeX
%%% TeX-master: "../Thesis"
%%% End:
