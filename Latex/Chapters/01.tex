% 1. Introduction
\section{Introduction}
\label{sec:introduction}

% Da sommario di Elisa
% - Prima parte sui neonati prematuri
% - Applicazione delle vie aeree per "pulire il fluido"
% - Definizione di strategie protettive

% % A cosa serve il liquido polmonare neonatale
% Durante la gravidanza le vie aeree del feto sono piene di liquido, il
% quale è definito liquido polmonare fetale, la cui presenza è
% fondamentale per lo sviluppo della larghezza delle vie aeree. Ne
% consegue che alla nascita il sistema respiratorio si debba svuotare di
% tale liquido per poter permettere l'entrata e l'uscita di aria,
% necessaria alla respirazione (processo di aerazione). Il
% riassorbimento inizia già qualche giorno prima del parto grazie a
% processi chimici dei canali sodio, e poi durante il parto naturale vi
% è fuoriuscita di liquido da bocca e naso grazie alla compressione del
% torace a cui viene sottoposto il neonato.

% % Differenza nell'areazione tra bambino a termine e pretermine.
% Se un bambino nasce a termine la percentuale di complicazione nel
% processo di aerazione è molto bassa. Totalmente differente è il caso
% di un bambino pretermine, ovvero di un bambino che nasce prima delle
% 37 settimane di gestazione a fronte delle 40 per una normale
% gravidanza.

% % Pressione positiva come trattamento
% L'applicazione di una pressione all'entrata delle vie aeree aiuta i
% bambini pretermine a rimuovere il fluido, e ad aprire gli alveoli
% chiusi. Una volta che gli alveoli sono stati reclutati serve meno
% pressione per mantenerli aperti e ventilare il polmone. Per tale
% ragione, effettuare strategie di reclutamento alla nascita ha il
% vantaggio di permettere l'inizio della ventilazione in un polmone più
% reclutato, l'utilizzo di pressioni ridotte, ottenendo così
% un'aerazione più omogenea del polmone e riducendo lo stress applicato
% ai tessuti. È importante sottolineare però, cha alla nascita il
% polmone è più delicato e quindi maggiormente esposto a danni, anche se
% vengono adottate procedure di ventilazione di routine.  Sviluppare
% quindi strategie protettive di reclutamento alla nascita potrebbe
% portare miglioramenti significativi in questo ambito.

% Anche se le manovre di reclutamento hanno acquistato maggiore
% interesse nell'ambito della ventilazione dei pretermine, non esiste
% ancora una strategia comune a livello medico. Procedure sperimentali
% vengono testate sugli animali, presentando difficoltà nell'ottenere
% dei risultati sia per l'invasività delle procedure sia per i problemi
% etici collegati. Modellizzazioni in silico del polmone adulto sono
% state utili per capire la patofisiologia e compiere diagnosi. Quindi,
% lo stesso approccio potrebbe aiutare ad analizzare le diverse
% strategie di reclutamento e il loro impatto sul polmone durante la
% prima aereazione alla nascita. Tuttavia, i modelli polmonari di
% neonati in-silico sono limitati alla descrizione fino alla prima
% generazione dell'albero bronchiale e, non si dimostrano quindi
% adeguati a simulare i cambiamenti fisiologici che avvengono alla
% nascita. 

% % Muovi in anatomical [primo]

% Queste considerazioni sono essenziali per sviluppare modelli accurati
% e utili per studiare la funzione polmonare nei neonati, specialmente
% in contesti clinici come la ventilazione oscillatoria.

% Perciò si è dovuto opportunamente modificare sia le caratteristiche
% anatomiche che quelle meccaniche dei tessuti per ottenere una
% modellizzazione coerente al caso del neonato nel dominio del
% tempo.

% % Spostare in mechanical (?) [primo]

% Quindi, sia il modello morfometrico che quello elettrico sono stati
% modificati per implementare i cambiamenti che avvengono alla nascita,
% i quali includono la modifica dei valori delle resistenze e induttanze
% che compongono i modelli elettrici delle vie aeree, in quanto
% dipendenti dalla viscosità e dalla densità del mezzo che li attraversa
% (aria o liquido polmonare fetale).

% % Muovi in mechanical [secondo]

% % Muovi in mechanical (?)
% Dei primi tentativi sono stati effettuati con «LTspice» e «CADENCE».
% Utilizzando strategie open-source per il modello, le licenze di
% software proprietario non sono più necessarie, rendendo il processo di
% sviluppo più accessibile.

% Il modello è stato sviluppato coerentemente con i dati presenti nella
% letteratura sia per quanto riguarda la geometria dell'albero sia per
% le caratteristiche meccaniche dei suoi tessuti, ottenendo simulazioni
% del processo di aerazione nel dominio del tempo, che permettono di
% confrontare diverse strategie di ventilazione.

% % Aim
% Lo scopo di questo studio è lo sviluppo di un modello in silico del
% polmone dei neonati per analizzare il processo di aerazione alla
% nascita.

During pregnancy, the fetal airways are filled with a fluid known as
fetal lung fluid, which is essential for the development of airway
width. Consequently, at birth, the respiratory system must expel this
fluid to allow air to enter and exit, a process necessary for
breathing (aeration). Fluid reabsorption begins a few days before
birth through chemical processes involving sodium channels, and during
natural childbirth, fluid is expelled from the mouth and nose due to
the compression of the neonate's chest. In full-term infants, the
likelihood of complications during aeration is very low. However, the
scenario is vastly different for preterm infants, who are born before
37 weeks of gestation compared to the typical 40 weeks of a normal
pregnancy.

Applying pressure at the entrance of the airways helps preterm infants
remove the fluid and open the closed alveoli. Once the alveoli are
recruited, less pressure is needed to keep them open and ventilate the
lung. Thus, employing recruitment strategies at birth has the
advantage of initiating ventilation in a more recruited lung, using
lower pressures, achieving more homogeneous lung aeration, and
reducing the stress applied to the tissues. It is important to note,
however, that at birth, the lung is more delicate and thus more
susceptible to damage, even with routine ventilation
procedures. Therefore, developing protective recruitment strategies at
birth could lead to significant improvements in this area.

Although recruitment maneuvers have gained more interest in preterm
ventilation, there is still no common medical strategy. Experimental
procedures are tested on animals, presenting challenges in obtaining
results due to the invasiveness of the procedures and associated
ethical issues\cite{al-jumaily2011,herrmann2016}. In silico modeling
of the adult lung has been useful for understanding pathophysiology
and making diagnoses. Thus, the same approach could help analyze
various recruitment strategies and their impact on the lung during
initial aeration at birth. However, in silico models of neonatal lungs
are limited to describing up to the first generation of the bronchial
tree and are therefore inadequate for simulating the physiological
changes that occur at birth.

These considerations are essential for developing accurate and useful
models for studying lung function in neonates, especially in clinical
contexts such as oscillatory ventilation. Consequently, both the
anatomical and mechanical characteristics of the tissues needed to be
appropriately modified to achieve a consistent model for the neonatal
case in the time domain. Therefore, both the morphometric and
electrical models were modified to implement the changes occurring at
birth, including the adjustment of resistance and inductance values in
the electrical models of the airways, as these depend on the viscosity
and density of the medium passing through them (air or fetal lung
fluid).

Initial attempts were made with "LTspice" and "CADENCE." Using
open-source strategies for the model eliminates the need for
proprietary software licenses, making the development process more
accessible. The model was developed in line with data from the
literature regarding both the geometry of the bronchial tree and the
mechanical characteristics of its tissues, resulting in time-domain
simulations of the aeration process, which allow for comparison of
different ventilation strategies\cite{mani2020}.


Aims of this projects are:
\begin{description}
\item Generate a model from neonatal CT scans to optimize the
  generation of airways, ensuring they adhere to the morphometric
  characteristics at various ages.
\item Develop an open-source mechanical model that allows for the
  simulation of mechanical properties along with fluid dynamics
  (capillary pressure associated with the interface).
\end{description}


% Commenti
% - Fragilità polmone neonatale
% - Differenze anatomiche tra neonato e adulto.
% - Albero bronchiale definizione etc. (penso anche ordini, branching etc).
% - Modello morfometrico definizione etc (?)
% - Modello polmone di agnello con capacità in parallelo.
% - Diametro trachea in funzione dell'eta gestazionale (grafico).

% \subsection{Premature newborns}
% \label{subsec:premature_newborns}

% Commenti
% Differenze tra sistema respiratorio nell'adulto e nel neonato e nel
% neonato prematuro.  (Slide 4 di [1]).

% \subsection{Morphometric Model}
% \label{subsec:morphometric_model}
% I am referring to \Cref{subsec:premature_newborns}

% \subsubsection{A brief definition}
% \label{subsubsec:morphometric_model_definition}

% Commenti
% Un'idea per questa definizione la si ritrova in
% `ModelloMorfometricoTriennale`.

% Commenti
% A morphometric model can be thought as a mathematical model composed
% by fixed parameters having both a physical and a geometrical (?)
% meaning.

% \subsubsection{Its structure}

% Slide 3 di [1]

% Riferimenti: `ModelloMorfometricoTriennale`,
% `ModelloMorfometricoTriennaleTesi`,
% `ModelloMorfometricoTriennaleBib/`


%%% Local Variables:
%%% mode: LaTeX
%%% TeX-master: "../Thesis"
%%% End:
