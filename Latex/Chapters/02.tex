% 2. Preterm Neonates
% 2.1. Challenges and Characteristics
% 2.2. Liquid Clearance
% 2.3. Analogies and Differences with Adult Lung

\section{Preterm Neonates}
\label{sec:preterm_neonates}

% LE FRAGILITÀ CAUSATE DALLA PREMATURITÀ
\subsection{Challenges and Characteristics}
\label{subsec:challenges_and_characteristics}

A baby born before reaching thirty-seven weeks of pregnancy is
considered preterm. Typically, a full-term pregnancy lasts about forty
weeks.  Preterm births can be further classified based on how early
the baby is born:

\begin{itemize}
\item Extremely preterm: Birth occurs before twenty-eight weeks.
\item Very preterm: Birth happens between twenty-eight and thirty-two weeks.
\item Moderate to late preterm: Birth occurs between thirty-two and thirty-seven weeks.
\end{itemize}

This issue should not be underestimated, as fifteen million babies are
born preterm each year\cite{who2013}.  Preterm infants often have
underdeveloped lungs.  Respiratory insufficiency in these babies can
be managed with ventilation support.

%% LIQUID CLEARANCE
\subsection{Liquid Clearance}
\label{subsec:liquid_clearance}

The respiratory transition from liquid to air in newborns involves
three stages:
\begin{enumerate}
\item No pulmonary gas exchange
\item Liquid clearance and initiation of gas exchange
\item Removal of residual liquid
\end{enumerate}

Initially, the baby cannot breathe because the lungs are still filled
with liquid. Two main factors influence the absorption of this liquid:
\begin{itemize}
\item \emph{Gestational Age}
\item \emph{Mode of Delivery} (either vaginal or cesarean section)
\end{itemize}

Before the baby can start breathing air, the liquid must be cleared
from the lungs. A significant contribution to this process comes from
a chemical mechanism: Na$^{\text{+}}$ reabsorption. The stress of
labor increases levels of adrenaline and vasopressin, which drive this
process. These hormones stimulate the airway epithelium to reverse the
osmotic gradient, leading to liquid uptake.

During birth, the liquid is expelled from the lungs through the nose
and mouth, a process known as the ``vaginal squeeze.'' This process
involves changes in fetal posture, such as increased spinal flexion,
which elevates the diaphragm and increases abdominal and
transpulmonary pressure.

The pressure gradient is crucial for driving air movements throughout
the lungs.

Transpulmonary pressure is the difference between alveolar pressure
and pleural pressure in the pleural cavity. If transpulmonary pressure
is zero, no air movement occurs. If it is positive or negative, air
flows out or in, respectively.

The presence of liquid in the interstitial compartment increases
pressure. As the alveolus empties and its pressure decreases, a small
amount of liquid re-enters, initiating a continuous cycle of liquid
clearance and re-entry in the alveolus.

%%% ANALOGIE E DIFFERENZE CON IL POLMONE ADULTO
\subsection{Analogies and Differences with Adult Lung}
\label{subsec:analogies_and_differences}

The respiratory system of a neonate presents both similarities and
differences compared to that of an adult.

A common aspect is the number of airways. By the end of gestation, the
branching of the bronchial tree is complete.

The differences lie in the size of the airways. The peripheral airways
in neonates, which are the smallest, are about half the size of those
in adults. However, the trachea in neonates is about one-third to
one-fourth the size of an adult's trachea, indicating no consistent
scaling rule.

The number of alveoli is also different. Infants have fewer alveoli
than adults, as their development and proliferation continue until
about eight years of age\cite{avery1973}.

Beyond anatomical differences, neonates also differ in mechanical
properties.

Neonates have higher airway resistance compared to adults, which
decreases continuously during the first year of life.

The thorax of neonates is highly compliant and deformable due to the
thin cartilage of the ribs and incomplete mineralization of the bones.

Thoracic compliance decreases during the first year of life.

Additionally, neonates' ribs are positioned more horizontally,
reducing the efficiency of the respiratory muscles.

\subsection{Resuscitation at Birth}
\label{subsec:resuscitation_at_birth}

The underdevelopment of the respiratory system and poor respiratory
drive in preterm infants often necessitate the use of positive
pressure to assist with lung aeration at birth. However, from the
first inflations, the preterm lung is particularly prone to injury. It
has been suggested that replicating the respiratory transition of term
infants by rapidly aerating the lung using an initial sustained
inflation (SI) may be beneficial for preterm infants. Nonetheless,
clinical and preclinical studies comparing SI to gradual lung aeration
using tidal inflations and positive end-expiratory pressure (PEEP)
have produced mixed results\cite{tingay2019}.

A clear understanding of how different aeration strategies affect
regional aeration, ventilation, and subsequent lung injury is
lacking. This gap in knowledge not only pertains to a critical
physiological event but also hinders efforts to improve care for
vulnerable infants at birth. Therefore, a comprehensive understanding
of the mechanisms involved in this process is crucial to developing
effective lung-protective ventilation strategies in the delivery
room\cite{veneroni2020}.

%%% Local Variables:
%%% mode: LaTeX
%%% TeX-master: "../Thesis"
%%% End:
