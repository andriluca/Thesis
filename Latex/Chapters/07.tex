% 7. Conclusion and Future Development

% 1. SOSTITUIRE XX CON DATI
% 2. LEGGI HORSFIELD1987 PER INSERIRE DEFINIZIONI RB, RD, RL.

\section{Conclusion and Future Development}

% As future development it is possible to edit properly airway
% generation parameters in order to better approximate the target
% morphometric characteristics, so to be able to produce a
% patient-specific anatomical model.  This allows for gathering more
% accurate mechanical parameters for the simulation phase.

Anatomical morphometric models of the adult lung have been extensively
developed and used in the literature to enhance our understanding of
pulmonary pathologies and to guide treatments. In contrast, anatomical
morphometric models of the newborn lung are largely absent. It is not
sufficient to simply scale down models developed for adult lungs to
fit newborn lungs. Newborn lungs are not merely smaller versions of
adult lungs; they exhibit significant differences in morphometric
characteristics, airway wall structure, and tissue composition and
properties. These differences must be considered when developing or
adapting mathematical models to accurately represent the functioning
of newborn lungs.

We developed an anatomical morphometric model of the newborn
lung. Using a CT scan of a newborn infant, we extracted the centreline
of XX generations and the lobe surfaces. We then reconstructed the
anatomy of the missing airways using a statistical algorithm
originally proposed for adult lungs, which we adapted for the newborn
lung. This algorithm assigned airway diameters based on proportions
measured in the newborn lung, providing several advantages over
previous approaches. Previous work, such as \textcite{mani2020},
rescaled adult models to match the diameters and lengths of the
trachea and terminal bronchioles in newborns. However, this method may
not accurately preserve the morphometric characteristics (Rb, Rd, and
Rl – see Chapter 1.XXX) of the entire airway tree. Our approach allows
for the direct setting of the desired Rd parameter in the model, which
differs between adults and newborns. Further analysis of the generated
tree can determine if all known morphometric characteristics of
newborn lungs (e.g., Rb and Rl) are respected or if optimization of
the arbitrary parameters initially set for adults (e.g., branching
length) is necessary.

We implemented a mechanical analog of the airway and acini in
Julia. This model accounts for changes related to aeration at birth,
allowing the simulation of the flow of fetal fluid towards the
periphery as air enters the airways. The model incorporates changes in
resistance (R) and compliance (I), as well as capillary pressure
developed in the airways at the fluid-air interface. Testing on a
subset of the anatomical tree yielded consistent results,
demonstrating the model's ability to simulate the phenomena involved
in lung aeration. Future developments will include simulating the
entire airway tree and analyzing the time required for full network
simulation.

This model enables open-source simulation of various aeration
strategies that can be applied at birth. Such simulations are crucial
for defining protective lung strategies that may reduce long-term
sequelae in preterm infants.

%%% Local Variables:
%%% mode: LaTeX
%%% TeX-master: "../Thesis"
%%% End:
