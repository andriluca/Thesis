\begin{figure}[H]\centering
  \begin{circuitikz}[scale=.9]
    % Circuit
    %% Main branch
    %%% Input Node
    \draw (1.5,0)
    node[ocirc] (circuit_IN) {}
    to[short, ] ++(1,0) coordinate(Z_IN)
    to[twoport, bipoles/twoport/width=1.0, t=$Z(2)$] ++(2,0) coordinate(C_g_IN)
    ;
    
    %% Main branch
    % \draw (Z_IN)
    % to[twoport, t=$Z_{2}$] ++(2,0) coordinate(C_g_IN)
    % ;

    %%% C_g branch
    \draw (C_g_IN)
    to[polar capacitor, invert, l_=$C_{\text{g}}(n)$, *-] ++(0,-3.5)
    node[ground]{} ++(0,0)
    ;

    %%% Zw branch
    \draw (C_g_IN)
    to[inductor, l=$I_{\text{t,i}}$] ++(2, 0)
    to[twoport, t={$\dfrac{G - j\cdot H}{\omega^\alpha}$},bipoles/twoport/width=2.0, bipoles/twoport/height=1.2] ++(3, 0)
    to[short] ++(0, -3.5)
    node[ground]{} ++(0,0)
    ;

  \end{circuitikz}
  \caption{An alveolar-tissue element is attached to the terminal
    airways in the tree. There is gas compression corresponding to
    volume of the acinus ($C_{\text{g}}$) and the tissue element is
    viscoelastic containing a tissue damping ($G$) coupled to
    elastance ($H$) to ensure a constant tissue hysteresis. $j$:
    imaginary unit, $I_{\text{t,i}}$: tissue
    inertance\cite{lutchen1997}.}
  \label{fig:acinus_impedance}

\end{figure}

%%% Local Variables:
%%% mode: LaTeX
%%% TeX-master: "../Thesis"
%%% End:
