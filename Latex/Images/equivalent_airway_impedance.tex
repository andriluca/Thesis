\begin{figure}[H]\centering
  \begin{circuitikz}[scale=.9]
    % Circuit
    %% Main branch
    %%% Input Node
    \draw (1.5,0)
    node[ocirc] (circuit_IN) {}
    to[short, ] ++(1,0) coordinate(R_IN)
    ;
    
    %% Main branch
    \draw (R_IN)
    to[short] ++(.5,0)
    to[resistor, l=$R(n) / 2$] ++(1,0)
    to[inductor, l=$I(n) / 2$] ++(3,0) coordinate (C_g_IN)
    to[short] ++(1.5,0) coordinate (SW_IN)
    to[resistor, l=$R(n) / 2$] ++(3,0)
    to[inductor, l=$I(n) / 2$] ++(1,0)
    to[short, -*] ++(1,0) coordinate(circuit_OUT)
    ;

    %%% C_g branch
    \draw (C_g_IN)
    to[polar capacitor, invert, l_=$C_{\text{g}}(n)$, *-] ++(0,-3.5)
    node[ground]{} ++(0,0)
    ;

    %%% Zw branch
    \draw (SW_IN)
    to[twoport, t={$Z_{w}(n)$},bipoles/twoport/height=1.2, *-] ++(0, -3.5)
    node[ground]{} ++(0,0)
    ;

    \draw (circuit_OUT)
    -- ++(0, .5)
    to ++(1.88,0)
    node[draw, anchor=west, thick] {$Z(n-1)$};
    
    \draw (circuit_OUT)
    -- ++(0, -.5)
    to ++(1,0)
    node[draw, anchor=west, thick] {$Z(n-1-\Delta$)};

    % Nodes
    %% IN
    \draw[->, >=stealth, thick] (0,0) -- (circuit_IN) node[midway, above] {Z(n)};
  \end{circuitikz}
  \caption{Impedance ($Z$) of a given order ($n$) of a single airway
    generation is calculated via an acoustic transmission line
    analysis, which accounts for shunting into gas compression in the
    tube ($C_{\text{g}}(n)$) and into nonrigid airway walls
    ($Z_{\text{w}}$)\cite{lutchen1997}.}
  \label{fig:airway_impedance}

\end{figure}

%%% Local Variables:
%%% mode: LaTeX
%%% TeX-master: "../Thesis"
%%% End:
