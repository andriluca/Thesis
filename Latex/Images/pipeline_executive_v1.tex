\begin{figure*}[]
  \centering
  \scalebox{.75}{
    \begin{tikzpicture}[]
      % Nodes
      %% Chaste & Morphometric Model
      \node (ct_acquisition)           [startstop] {CT Acquisition};
      %%% Upper branch
      \node (lobes_segmentation)       [process, right of=ct_acquisition, xshift=8.25cm, yshift=+1cm]      {Lobes segmentation};
      \node (airways_segmentation)     [process, below of=lobes_segmentation, xshift=-2.5cm, yshift=-1cm]  {Airways segmentation};
      \node (centreline_extraction)    [process, below of=lobes_segmentation, xshift=+2.5cm, yshift=-1cm]  {Centreline extraction};
      \node (anatomical_generation)    [process, minimum width=5cm, text width=5cm, right of=ct_acquisition, xshift=16.5cm]                  {Algorithmic Generation of Distal Airway Centrelines};
      %% Julia & Mechanical Model
      \node (airways_model)            [startstop, below of=ct_acquisition, yshift=-5cm]               {Airways model};
      \node (acini_model)              [startstop, below of=airways_model, yshift=-1cm]                  {Acini model};
      \node (model_instantiation)      [process, below of=airways_segmentation, yshift=-5cm]                     {Mechanical model instantiation};
      \node (DAE_solution)             [process, below of=centreline_extraction, yshift=-5cm]               {DAE System solution};
      \node (tracheal_pressure)        [startstop, above of=DAE_solution, yshift=2.5cm]                      {Tracheal pressure waveform};
      \node (simulations)              [startstop, below of=anatomical_generation, yshift=-6cm]                   {Simulations results};
      \node (phi_length)               [startstop, minimum width=5cm, text width=4.5cm, above of=model_instantiation, yshift=2.5cm]                 {Diameters, length, father airway of each airway};
      
      
      % % Arrows
      \draw [arrow] (ct_acquisition.east)                              -- ++(1.5,0) coordinate(tmp)     |- (airways_segmentation.west);
      \draw [arrow] (airways_segmentation)                             -- (centreline_extraction);
      \draw [arrow] (centreline_extraction.east)                       -- ++(.5,0) coordinate(tmp1)    |- (anatomical_generation.west);
      \draw [arrow] (model_instantiation)                              -- (DAE_solution);
      \draw [arrow] (DAE_solution)                                     -- (simulations);
      \draw [arrow] (tmp)                                              |-   (lobes_segmentation.west);
      \draw [arrow] (lobes_segmentation)                               --   (tmp1|-lobes_segmentation) |- (anatomical_generation.west);
      \draw [arrow] (phi_length.south)  -- (model_instantiation.north);
      \draw [arrow] (acini_model.east) -- ++(1.5,0)  |- (model_instantiation.west);
      \draw [arrow] (tracheal_pressure.south) -- (DAE_solution.north);
      \draw [arrow] (airways_model.east) -- ++(1.5,0)  |- (model_instantiation.west);
      \draw [dashed, thick, ->,>=stealth] (12.5,-1.775) -- ++(0,-.50)                   -| (phi_length.north);

      
      % % Frame around a part of the flowchart
      \begin{pgfonlayer}{background}
        \node[rounded corners=3mm,
        draw=blue,
        thick, dashed,
        % fit=(airways_segmentation)(centreline_extraction)(img_lobes_segmentation)(img_anatomical_generation),
        fit=(airways_segmentation)(centreline_extraction)(lobes_segmentation)(anatomical_generation),
        fill=cyan!5,
        inner sep=7pt,
        label={[anchor=south]above:\textsc{\textcolor{blue}{Morphometric model}}}] {};
      \end{pgfonlayer}
      \begin{pgfonlayer}{background}
        \node[rounded corners=3mm,
        draw=red,
        thick, dashed,
        fit=(airways_model)(acini_model)(model_instantiation)(DAE_solution),
        fill=magenta!5,
        inner sep=7pt,
        label={[anchor=north]below:\textsc{\textcolor{red}{Mechanical model}}}] {};
      \end{pgfonlayer}
    \end{tikzpicture}
  }
  \caption{Data pipeline.  The process begins with a
    \emph{patient-specific image} (i.e. CT) of a premature newborn.
    The extracted data, comprising \emph{two segmentations}, are then
    processed to obtain an anatomical surrogate of the airway tree.
    This is necessary due to scanner resolution not allowing for the
    discrimination and localization of small branches.  From the
    resulting morphometric model, the \emph{mechanical parameters} can
    be derived, which are essential for generating an accurate
    simulation model.  Finally, a numerical solver for differential
    equations provides the final output.}
  \label{fig:data_pipeline}
\end{figure*}

%%% Local Variables:
%%% mode: LaTeX
%%% TeX-master: "../Executive"
%%% End:
